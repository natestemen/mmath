\documentclass[12pt]{article}

\setlength{\headheight}{14.5pt}
\usepackage[utf8]{inputenc}
\usepackage[margin=1.2in]{geometry}
\usepackage{microtype}
\usepackage{fancyhdr}
\usepackage{palatino}
 
\pagestyle{fancy}
\fancyhf{}
\lhead{Cover Letter}
\rhead{D-Wave Intern}

\renewcommand{\baselinestretch}{1.42}
\setlength\parskip{0.5em}

\begin{document}
\noindent
To whom it may concern,

I am a second year PhD student at the Institute for Quantum Computing (IQC) who is interested in making quantum technologies a reality.
I began my work at IQC with Professor Joel Wallman working on quantum error correcting techniques and applications of Lie theory to quantum information.
At D-Wave Systems I'd like to apply my quantum information knowledge, combined with my programming skills to work on the real problems we face in constructing quantum computers.

Having spent 4 years as a software developer at Overleaf, I developed strong programming abilities and learned how to be a core member of a fast moving team.
From finding and diagnosing bugs, to doing user research and shaping new features, I've run the gamut of the responsibilities a software developer may have.
My programming experience has lent me a new lens through which I view problems, and will allow me to work on software projects with less startup time

During my undergraduate I spent two summers in the Wright Lab at Yale where I got my hands dirty running experiments testing components of our detector to optimize for our goal of searching for a new type of neutrino.
I also spent a summer in the mathematics department at NYU where I numerically computed solutions to partial differential equations modelling electromagnetic pulses travelling through a medium.
These summers gave me the understanding of how progress in science is made, and allowed me to experience both theoretical and experimental sciences.

With the advances transpiring in quantum technology I believe now to be an ideal time to be involved in creating a technology that will bring us into a new computing era.
The many serious political, social, and environmental effects of quantum technologies require it's creators to be both well versed, and well rounded quantum scientists.
I believe the internship opportunity at D-Wave to be an ideal place to expose me to the problems we face in building quantum computers and will help me become the well rounded researcher I aspire to be.

\noindent
Best,

\parskip=0pt plus 1pt
Nathaniel T. Stemen
\end{document}