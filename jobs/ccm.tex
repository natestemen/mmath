\documentclass[12pt]{article}

\setlength{\headheight}{14.5pt}
\usepackage[utf8]{inputenc}
\usepackage[margin=1.2in]{geometry}
\usepackage{microtype}
\usepackage{fancyhdr}
\usepackage{palatino}
 
\pagestyle{fancy}
\fancyhf{}
\lhead{Cover Letter}
\rhead{CCM Summer Research Associate}

\renewcommand{\baselinestretch}{1.25}
\setlength\parskip{0.5em}

\begin{document}
\noindent
To whom it may concern,

I am a second year PhD student at the University of Waterloo's Institute for Quantum Computing who is interested in making quantum technologies a reality.
I began my work at IQC with Professor Joel Wallman working on quantum error correcting techniques and applications of Lie theory to quantum information.
At the Center for Computational Mathematics I'd like to apply my mathematical and programming knowledge to better understand machine learning, and broaden my mathematical expertise.
My studies, and expressed interest at CCM may seem misaligned, but with the increasing applications and use of machine learning in quantum, I expect a more theoretical understanding of machine learning to benefit quantum sciences greatly.

Having spent 4 years as a software developer at Overleaf, I developed strong programming abilities and learned how to be a core member of a fast moving team.
From finding and diagnosing bugs, to doing user research and shaping new features, I've run the gamut of the responsibilities a software developer may have.
My programming experience has lent me a new lens through which I view problems, and will allow me to work on software projects with less startup time

During my undergraduate I spent two summers in the Wright Lab at Yale where I got my hands dirty running experiments testing components of our detector to optimize for our goal of searching for a new type of neutrino.
I also spent a summer in the mathematics department at NYU where I numerically computed solutions to partial differential equations modelling electromagnetic pulses travelling through a medium.
These summers gave me the understanding of how progress in science is made, and allowed me to experience theoretical, experimental, and computational sciences.

As the ubiquity of problems benefiting from a computational understanding grows, it is important that solution architects are well versed in the problems specialization, as well as versatile; applying cross discipline techniques to aid in a solution.
I believe an internship opportunity at the CCM to be an ideal place to expose me to new computational problems, provide me with new tools to push forward research, and help me become the well rounded researcher I aspire to be.

\noindent
Best,

\parskip=0pt plus 1pt
Nathaniel T. Stemen
\end{document}