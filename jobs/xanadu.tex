\documentclass[12pt]{article}

\setlength{\headheight}{14.5pt}
\usepackage[utf8]{inputenc}
\usepackage[margin=1.2in]{geometry}
\usepackage{microtype}
\usepackage{fancyhdr}
\usepackage{palatino}
 
\pagestyle{fancy}
\fancyhf{}
\lhead{Nathaniel T. Stemen}
\rhead{Xanadu Residency Program}

\rfoot{Statement of Interest}

\renewcommand{\baselinestretch}{1.2}

\begin{document}
\noindent
To Whom It May Concern,

\noindent
With the advances transpiring in quantum technology I believe now to be an ideal time to be involved in creating a technology that will bring us into a new computing era.
Because there are many serious political, social, and environmental effects of quantum technologies it is of the utmost importance that those developing these technologies have a well rounded and balanced understanding of the environment.
It is for this reason that I believe working at Xanadu would allow me to take my largely theoretical quantum information knowledge, and apply it to better understand the real problems we face in constructing this technology.

Having spent 3 years as a software developer at Overleaf I have developed both strong programming abilities, as well as learned how to be a core member of a fast moving team.
Working at Overleaf has shown me how enjoyable programming can be, and has lent me a new lens through which I view problems.
More recently---while completing my PhD---I've switched from a software development role at Overleaf to that of a content developer.
In this role my work is focused on writing clear and concise documentation and tutorials on how to use Overleaf.
This work has allowed me to work on my writing and communication skills while building pedagogical strategies for effectively teaching \LaTeX{} to newcomers and experts alike.
This work has only improved as I've complete the University of Waterloo's \emph{Certificate of Teaching Excellence} which I've completed as of this month.

My theoretical understanding of quantum sciences comes first from NYU where I developed a strong understanding of both the mathematics of quantum information sciences, and the physics of quantum mechanics.
While at NYU I also had 3 summer research experiences where I performed varied tasks from running detector uniformity tests in a physics lab, to trying to analytically prove the existence of solutions of a certain partial differential equation.
In 2020 I began my PhD at the Institute for Quantum Computing where I worked to apply my understanding of quantum mechanics to pick up the new language of quantum information.
I also began work with Professor Joel Wallman on quantum error correcting technologies.

At Xanadu I'd like to use my programming experience, combined with my knowledge in quantum information to work on quantum software projects like Pennylane.
The particular project of compiling Gaussian subcircuits is also of interest and circuit compilation is something I've found fascinating since hearing members of Joel's group discuss it at our group meetings.
I'd also love to utilize my experience writing technical content to work with Josh Izaac and/or Olivia Di Matteo developing tutorials and online courses that take advantage of modern web technologies.

\noindent
Best,

Nathaniel T. Stemen
\end{document}