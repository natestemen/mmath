\documentclass[12pt]{article}

\setlength{\headheight}{14.5pt}
\usepackage[utf8]{inputenc}
\usepackage[margin=1.2in]{geometry}
\usepackage{microtype}
\usepackage{fancyhdr}
\usepackage{palatino}
 
\pagestyle{fancy}
\fancyhf{}
\lhead{Statement of Interest}
\rhead{Xanadu Residency Program}

\renewcommand{\baselinestretch}{1.14}
\setlength\parskip{0.5em}

\begin{document}
\noindent
To whom it may concern,

I am a second year PhD student at the Institute for Quantum Computing (IQC) who is interested in helping make quantum technologies a reality.
I began my work at IQC with Professor Joel Wallman working on quantum error correcting techniques and mathematical aspects of quantum information.
At Xanadu I'd like to apply my quantum information knowledge, combined with my programming skills to work on quantum software projects like Pennylane and compiling Gaussian subcircuits.
I also have a background writing technical content and would be interested in developing tutorials and online content with Josh Izaac and Olivia Di Matteo.

Having spent 3 years as a software developer at Overleaf, I developed strong programming abilities and learned how to be a core member of a fast moving team.
Working at Overleaf has shown me how enjoyable programming can be, and has lent me a new lens through which I view problems.
While completing my PhD, I've taken on the role of content developer where I write clear and concise documentation and tutorials on how to use Overleaf.
This work has only improved as I've completed the University of Waterloo's \emph{Fundamentals of University Teaching Program} which helps graduate students develop evidence-based teaching strategies.
I believe my work at Overleaf will allow me to work on software projects without much startup time, and prepares me well to create pedagogically effective quantum oriented content.

My theoretical understanding of quantum sciences comes first from NYU where I developed a strong understanding of both the mathematics of quantum information, and the physics of quantum mechanics.
While at NYU I also had 3 summer research experiences where I performed a variety of tasks: running detector uniformity tests in a physics lab and trying to analytically prove the existence of solutions of a partial differential equation in the mathematics department are two disparate examples.
This work has allowed me to understand how progress in science is made, and given me a range of tools to attack problems with.

With the advances transpiring in quantum technology I believe now to be an ideal time to be involved in creating a technology that will bring us into a new computing era.
The many serious political, social, and environmental effects of quantum technologies require it's creators to be both well versed, and well rounded quantum scientists.
I believe the Xanadu Residency Program to be an ideal place to expose me to the real problems we face in building quantum computers and will help me become the well rounded researcher I aspire to be.

\noindent
Best,

\parskip=0pt plus 1pt
Nathaniel T. Stemen
\end{document}