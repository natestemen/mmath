\documentclass[12pt]{article}

\setlength{\headheight}{14.5pt}
\usepackage[utf8]{inputenc}
\usepackage[margin=1.2in]{geometry}
\usepackage{microtype}
\usepackage{fancyhdr}
\usepackage{palatino}
 
\pagestyle{fancy}
\fancyhf{}
\lhead{Nathaniel T. Stemen}
\rhead{Xanadu Residency Program}

\rfoot{Statement of Interest}

\renewcommand{\baselinestretch}{1.2}

\begin{document}
\noindent
To Whom It May Concern,

With the advances transpiring in quantum technology I believe now to be an ideal time to be involved in creating a technology that will bring us into a new computing era.
Because there are many serious political, social, and environmental effects of quantum technologies it is of the utmost importance that creators of these technologies have a well rounded and balanced understanding of the landscape.
It is for this reason that I believe working at Xanadu would allow me to take my theoretical knowledge in quantum information and apply it to better understand the real problems we face in constructing quantum computers.

Having spent 3 years as a software developer at Overleaf I developed strong programming abilities and learned how to be a core member of a fast moving team.
Working at Overleaf has shown me how enjoyable programming can be, and has lent me a new lens through which I view problems.
More recently---while completing my PhD---I've switched from a software development role at Overleaf to that of a content developer.
In this role my work is focused on writing clear and concise documentation and tutorials on how to use Overleaf while building effective pedagogies for teaching \LaTeX{} to newcomers and experts alike.
This work has only improved as I've completed the University of Waterloo's \emph{Certificate of Teaching Excellence}.

My theoretical understanding of quantum sciences comes first from NYU where I developed a strong understanding of both the mathematics of quantum information sciences, and the physics of quantum mechanics.
While at NYU I also had 3 summer research experiences where I performed varied tasks from running detector uniformity tests in a physics lab, to trying to analytically prove the existence of solutions of a partial differential equation in the mathematics department.
In 2020 I began my PhD at the Institute for Quantum Computing where I began work with Joel Wallman on quantum error correcting techniques.

At Xanadu I'd like to use my programming experience, combined with my knowledge in quantum information to work on quantum software projects like Pennylane.
The particular project of compiling Gaussian subcircuits is of interest and circuit compilation generally is something I've found fascinating since hearing about it from members at IQC.
I'd also love to utilize my experience writing technical content to work with Josh Izaac/Olivia Di Matteo developing tutorials and online courses that take advantage of modern web technologies.
I've seen some great quantum content on the web, but there's certainly more that can be done in the realm of interactivity.

\noindent
Best,

Nathaniel T. Stemen
\end{document}