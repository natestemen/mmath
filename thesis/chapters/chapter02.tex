%*****************************************
\chapter{Examples}\label{ch:examples}
%*****************************************

\section{Quantum Circuits}
There are three main models used for quantum computation:
\begin{itemize}
    \item circuital quantum computing
    \item adiabatic quantum computing
    \item measurement based quantum computing
\end{itemize}
Because of the ease in implementing universal quantum computation, the circuital model has become the most popular. % TODO
In this model we represent programs diagrammatically.
\begin{figure}[h]
    \centering
    \begin{quantikz}
        & \gate{U_0} & \ctrl{1} & \gate{U_1} & \ctrl{1}            & \swap{2} & \gate[wires=3]{U_3} & \qw \\
        & \gate{U_0} & \targ{}  & \octrl{-1} & \gate[wires=2]{U_2} & \qw      &                     & \qw \\
        & \gate{U_0} & \qw      & \qw        &                     & \targX{} &                     & \qw
    \end{quantikz}
    \caption{Example Quantum Circuit}
    \label{fig:excircuit}
\end{figure}
The theoretical laws of quantum mechanics tell us that time evolution is governed by unitary operators.
Hence in theoretical developments of quantum circuits we

\section{Formalizing}
The first important object to define is that of a \textbf{quantum gate set}.
These are the gates that we will be able to natively perform on our hardware.
\begin{definition}
    A \emph{quantum gate set} is a (typically finite) subset $G\subseteq \U{2^n}$.
\end{definition}
From these gates, we can construct a \textbf{quantum circuit} by applying a sequence of elements from the gate set.
\begin{definition}
    Let $G$ be a quantum gate set.
    A \emph{quantum circuit} is a an element of the free group $\free{G}$.
\end{definition}
Thus if our gate set $G = \qty{a, b, c}$, then a circuit can be any word from the set $\qty{a, a^{-1}, b, b^{-1}, c, c^{-1}}$.
Here we must include inverses because if you can perform a gate $U$, you can always perform it's inverse as well. % TODO: why???
Possible words are $abcaac^{-1}$, $bbb^{-1}ca^{-1}$, $c^{-1}b^{-1}a$, and so on.


Take $\mult: \free{G} \to U^{2^n}$ to be the map defined by $\mult(C) = \mult(U_1U_2\cdots U_n) = U_1\cdot U_2\cdots U_n$.

With these two definitions we can already formulate a very important question.
Suppose we have a target unitary $U\in\U{2^n}$.
Is there a circuit $C\in\free{G}$ such that $\dbrackets{C} = U$?
hi
If we'd like our circuit to implement a target unitary $U\in\U{2^n}$,
% In \cite{optimalSyn}