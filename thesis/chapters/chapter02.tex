%*****************************************
\chapter{Examples}\label{ch:examples}
%*****************************************

\section{Quantum Circuits}
There are three main models used for quantum computation:
\begin{itemize}
    \item circuital quantum computing
    \item adiabatic quantum computing
    \item measurement based quantum computing
\end{itemize}
Because of the ease in implementing universal quantum computation, the circuital model has become the most popular. % TODO
In this model we represent programs diagrammatically.
\begin{figure}[ht]
    \centering
    \begin{quantikz}
        & \gate{U_0} & \ctrl{1} & \gate{U_1} & \ctrl{1}            & \swap{2} & \gate[wires=3]{U_3} & \qw \\
        & \gate{U_0} & \targ{}  & \octrl{-1} & \gate[wires=2]{U_2} & \qw      &                     & \qw \\
        & \gate{U_0} & \qw      & \qw        &                     & \targX{} &                     & \qw
    \end{quantikz}
    \caption{Example Quantum Circuit}\label{fig:excircuit}
\end{figure}
The theoretical laws of quantum mechanics tell us that time evolution is governed by unitary operators.
Hence in theoretical developments of quantum circuits we

\section{Formalizing}
The first important object to define is that of a \textbf{quantum gate set}.
These are the gates that we will be able to natively perform on our hardware.
\begin{definition}
    A \emph{quantum gate set} is a (typically finite) subset $G\subseteq \U{2^n}$. An element of $G$ is called a quantum gate.
\end{definition}
From these gates, we can construct a \textbf{quantum circuit} by applying a sequence of elements from the gate set.
\begin{definition}
    Let $G$ be a quantum gate set, and let $G^*$ denote\footnote{This operation is known as the Kleene star.} the set of words over $G$.
    A \emph{quantum circuit} is an element of $G^*$.
\end{definition}
Thus if our gate set $G = \qty{a, b, c}$, then some example circuits may be $aacba$, $cccbbb$, $cbbcbab$.
Something to note here is that in this abstraction, all of our quantum gates are assumed to act on all qubits.
Hence if we have hardware with 2 qubits, and we can perform a Pauli $X$ gate on either qubit, then our gate is not simply $\qty{X}$, but rather $\qty{\1\otimes \1, \1\otimes X, X\otimes \1, X\otimes X}$.
Sometimes this gate set is denoted $\qty{\1, X_0, X_1, X_0X_1}$, but we will try and be explicit here.

We can now define a map $\mult: G^* \to \U{2^n}$ which takes a quantum circuit, or sequence of gates, and multiplies them together to get a final unitary: $\mult(U_1U_2\cdots U_n) = U_1\cdot U_2\cdots U_n$.
This map allows us to frame the following important question.

\begin{question}\label{qu:synthesis}
    Given a quantum gate set $G$, and unitary $U\in\U{2^n}$, does there exist a circuit $C\in G^*$, such that $\mult(C) = U$?
\end{question}
In the case this can we done, we say that a circuit $C$ implements a unitary $U$.
Further, if the answer to~\ref{qu:synthesis} is positive, there is often a follow on question.
\begin{question}\label{qu:optimalsynthesis}
    If $\mult(C) = U$, and $f: G^* \to \R$ is a cost function, can we find
    \begin{equation*}
        C_\text{min} = \argmin_{C\in G^*} \qty{f(C) : \mult(C) = U}?
    \end{equation*}
\end{question}
Some examples of common cost functions are given below, and multiple can be used in the case of tie-breaking.
\begin{itemize}
    \item $f(C) = \mathtt{length}(C)$ (commonly referred to as the depth of the circuit)
    \item $f(C) = $ \# of uses of a particular gate in $C$
    \item $f(C) = \mathtt{duration}(C)$
\end{itemize}

\begin{definition}
    A \emph{quantum chip's topology} is a graph $G = (V, E)$ with the vertices representing qubits, and edges representing connections between qubits where quantum gates of the correct arity can be applied.
\end{definition}

As an example take the the topology of IBM's \texttt{ibmq\_jakarta} shown in~\ref{fig:ibm-jakarta}.
``3'' being connected to ``1'' means that we can apply a 2-qubit unitary targeting both of those qubits, however the hardware does not support native 2-qubit gates between qubits ``2'' and ``6''.
\begin{figure}[ht]
    \centering
    \begin{tikzpicture}
        \tikzset{
            main node/.style={
                    very thick,
                    circle,
                    fill=blue!20,
                    draw,
                    minimum size=1cm,
                    inner sep=0pt
                }
        }
        \node[main node] (0) {$0$};
        \node[main node] (1) [right = 1cm of 0] {$1$};
        \node[main node] (2) [right = 1cm of 1] {$2$};
        \node[main node] (3) [below = 1cm of 1] {$3$};
        \node[main node] (5) [below = 1cm of 3] {$5$};
        \node[main node] (4) [left  = 1cm of 5] {$4$};
        \node[main node] (6) [right = 1cm of 5] {$6$};

        \path[draw, very thick]
        (0) edge node {} (1)
        (1) edge node {} (2)
        (1) edge node {} (3)
        (3) edge node {} (5)
        (5) edge node {} (4)
        (5) edge node {} (6);
    \end{tikzpicture}
    \caption{IBMQ Jakarta Architecture}\label{fig:ibm-jakarta}
\end{figure}

We can now define the main problem of quantum circuit compilation: that of the qubit mapping problem.\footnote{This sometimes also goes by the name of the qubit routing problem, or qubit scheduling problem although sometimes these mean slightly different things.}
If we have both a quantum circuit $C\in G^*$, and a quantum computer with network topology $G$, can the computer perform our circuit?
In order to address this question we must first talk about universal gate sets.
\subsection{Universal Gate Sets}
In order to harness the full power of a quantum computer, we hope it to be able
\begin{definition}
    A gate set $G$ on $n$ qubits is called \emph{universal} if for all $U\in\U{2^n}$ there exists a circuit $C\in G^*$ such that $\mult(C) = U$.
\end{definition}

The question of which gate sets are universal for quantum computation is important both for our theoretical understanding of quantum computation, but also for building physical devices.
Examples that have been shown to be universal are
\begin{itemize}
    \item \CNOT{} plus $\U{2}$ as shown in~\cite{universal-cnot-u2}
    \item \CNOT{}, Hadamard, and the $\frac{\pi}{8}$-gate\footnote{The $\frac{\pi}{8}$-gate is also sometimes called $T$ and has matrix representation $\smqty[1 & 0 \\ 0 & \e^{\iu\pi/4}]$.} as shown in~\cite{universal-cnot-had-p8}
    \item \CCNOT{} (Toffoli), Hadamard, and the $\frac{\pi}{4}$-gate\footnote{The $\frac{\pi}{4}$-gate is also sometimes called $S$ and has matrix representation $\smqty[1 & 0 \\ 0 & \e^{\iu\pi/2}]$.} as shown in~\cite{universal-ccnot-had-p4}
    \item \CNOT{} plus any single qubit gate that does not preserve the computational basis and is not the Hadamard gate as shown in~\cite{universal-cnot-basis-change}
\end{itemize}


% With these two definitions we can already formulate a very important question.
% Suppose we have a target unitary $U\in\U{2^n}$.
% Is there a circuit $C\in\free{G}$ such that $\dbrackets{C} = U$?
% hi
% If we'd like our circuit to implement a target unitary $U\in\U{2^n}$,
% In \cite{optimalSyn}