%*****************************************
\chapter{Quantum Computation}\label{ch:computation}
%*****************************************

In this chapter we will lay the groundwork for the necessary ideas from quantum computation.
We will not attempt to introduce quantum computation from the ground up, but instead introduce and emphasize some ideas we will focus on first.
The notation used here will mostly follow~\cite{watroustqi} and we recommend~\cite{nielsenchuang} for a more thorough introduction to the material.

\section{Historical Development}\label{sec:history}
Without a precise definition of quantum computing it's hard to give a precise storyline through the subject.
However, ideas rarely have clear cut boundaries and we must push forward to understand our messy world regardless.

One of the core tenets of quantum theory is that, at this scale, nature is reversible.
Hence when physicist Charles H. Bennett began investigating reversible Turing machines in~\cite{reversibleturing} we might say the field of quantum computing was \emph{just} beginning to start.
Since Turing machines are the mathematical and theoretical foundation for modern computers, it makes sense that a reversible Turing machine might lay the groundwork as the foundation for a computer that operators under quantum mechanical law.
More than 6 years later, Paul Benioff extended this work to describe a fully quantum mechanical version of a Turing machine in his paper \citetitle{quantumturing}~\cite{quantumturing}.

Once the theoretical foundation had been laid by Bennett and Benioff, Richard Feynman brought the idea mainstream when he proposed using these new computers to simulate quantum mechanics itself.
This idea was very attractive at the time (1981) since our classical computers were not powerful enough to simulate large quantum systems,\footnote{In fact, they still aren't!} and since Feynman was such a popular figure the idea finally took hold.
Feynman motivated the need for a new paradigm in computing as such:
\begin{quote}
    Nature isn't classical, dammit, and if you want to make a simulation of nature, you'd better make it quantum mechanical, and by golly it's a wonderful problem, because it doesn't look so easy.
    \attrib{Richard P. Feynman~\cite{feynmansimulator}}
\end{quote}

Even with one of the most famous physicists popularizing the idea, it took another 10 years to see the next major development which came in~\cite{deutch-jozsa-algo} where David Deutsch and Richard Jozsa gave an example of a problem that is solved exponentially faster on a quantum computer than a classical one.
If there was any hesitancy from the academic community at this point about the theoretical usefulness of a quantum computer, this result showed real potential for the emerging technology.
More applications start rolling in with quantum teleportation~\cite{quantumteleportation} and famously Peter Shor's polynomial time algorithm to factor integers (and hence break many modern cryptosystems)~\cite{shor-encryption}.

The latter caught the eye of the US Government and within the year of Shor's publication the \ac{NIST} organized the first government funded conference on quantum computation.\footnote{It's likely this is when quantum computation was put on the radar of the US government. In 2014 leaked documents showed the National Security Agency had begun a project dubbed ``Owning The Net'' whose purpose was to use a quantum computer to break internet cryptography and to ``gain access to and securely return high value target communications''. The status of the project---which also goes under the moniker ``Penetrating Hard Targets''---is unknown.}
There have been many major milestones since this, but perhaps most important to note is the first experimental realization of a qubit happened in~\citeyear{firstqubit} by Yasunobu Nakamura and Jaw-Shen Tsai~\cite{firstqubit}.
Hence it was less than 20 years from when Feynman demonstrated the potential usefulness of a quantum computer to the first experimental realization of the idea.

Since then ambitions have risen and technological progress has allowed for more and more qubits and quantum computers today have even been shown to complete tasks that classical ones cannot in any feasible amount of time.
In particular a team at China's Hefei National Laboratory used their 66-qubit computer\footnote{Affectionately named Zuchongzhi after Chinese mathematican Zu Chongzhi whose computation of $\pi$ was more accurate than any other for more than 800 years.} to complete a task in 4 hours that would take the most performant classical computers tens of thousands of years~\cite{zuchongzhi}.

More recently John Preskill has coined the term \ac{NISQ} as a characterization of the quantum computers that have dominated the past decade, and will likely continue to for the next few years~\cite{nisq}.
He takes these to be computers with 50--100 qubits for which noise will be a major factor in deciding what quantum circuits we can and cannot run.
The problem presented in this document is relevant to quantum computers past the \ac{NISQ}-era, but are especially important as we attempt to squeeze every ounce of computation out of them.

% TODO conclusion

\section{Quantum Computation}

In this section we will go over the basics of quantum computation.
% This section is written with a moderate amount of protest since I do not believe I will give a pedagogically proper and sound treatment of the material. % TODO make this a little lighter
Before continuing I would like to recommend~\cite{nielsenchuang} as well as \url{https://quantum.country} as great resources to learn the basics of quantum computing.

% TODO: talk about these
% There are three main models used for quantum computation:
% \begin{itemize}
%     \item circuital quantum computing
%     \item adiabatic quantum computing
%     \item measurement based quantum computing
% \end{itemize}
% Because of the ease in implementing universal quantum computation, the circuital model has become the most popular. % TODO

\subsection{Formalism}

A quantum bit, or \textbf{qubit} for short, is a vector $\ket{\psi}$ in 2-dimensional complex space $\C^2$ such that $\norm{\ket{\psi}} = 1$.
Often the following canonical basis is chosen and referred to as the computational basis.
\begin{align}
    \ket{0} \defeq \mqty[1 \\ 0] & & \ket{1} \defeq \mqty[0 \\ 1]
\end{align}
In this basis a qubit is a vector
\begin{equation}\label{eq:qubit}
    \ket{\psi} = \alpha\ket{0} + \beta\ket{1} = \mqty[\alpha \\ \beta]
\end{equation}
with the normalization condition that $\abs{\alpha}^2 + \abs{\beta}^2 = 1$.
In the case of \cref{eq:qubit} the state $\ket{\psi}$ is said to be in a \textbf{superposition} of state $\ket{0}$ and $\ket{1}$.

We often need to understand more complicated systems than just simple qubits, and to do so we use the \textbf{tensor product} to build up systems from subsystems.
\Eg{} if $\ket{\psi}\in\C^2$ and $\ket{\phi}\in\C^2$ represent two distinct physical qubits, we can represent the combined system as a single vector $\ket{\psi}\otimes \ket{\phi}$ in a larger complex Euclidean space $\C^2\otimes \C^2\cong \C^4$.
In the computational basis we can expand this as
\begin{align}
    \ket{\psi}\otimes \ket{\phi} & = \qty(\alpha\ket{0} + \beta\ket{1})\otimes\qty(\gamma\ket{0} + \delta\ket{1})                                                                \\
                                 & = \alpha\gamma\ket{0}\otimes\ket{0} + \alpha\delta\ket{0}\otimes\ket{1} + \beta\gamma\ket{1}\otimes\ket{0} + \beta\delta\ket{1}\otimes\ket{1}
\end{align}
where $\alpha, \beta, \gamma, \delta \in \C$.

With the objects of the theory defined, we must now understand the dynamics, or choreography of the theory.
As stated in~\cref{sec:history}, we take the theory of quantum mechanics to be reversible, and hence any operation we perform on a qubit $\ket{\psi}$ must be undo-able.
Thankfully linear algebra has just the tool to transform complex vectors in a reversible, and general way: unitary matrices!
\begin{definition}
    An $n\times n$ matrix $A$ is called \emph{unitary} if
    \begin{equation}\label{eq:unitary}
        AA^\dagger = A^\dagger A = \1
    \end{equation}
    where $^\dagger$ is the conjugate transpose.
    The collection of unitary matrices of dimension $n$ is called the \emph{unitary group} and is denoted \gls{un}.
\end{definition}
Hence when we have a qubit $\ket{\psi}$ and perform some action on it, we then have a new state $\ket{\phi} = U\ket{\psi}$ where $U$ represents whatever action we performed.
The condition shown in~\cref{eq:unitary} is quite restrictive: where a general $n \times n$ matrix has $2n^2$ real degrees of freedom, an element of $\U{n}$ only has $n^2$.\footnote{This is to say $\dim_\R\U{n} = n^2$.}
In fact for a general element of $\U{2}$ we can decompose it into pieces that look much more familiar.
\begin{example}
    Let $A$ be an arbitrary element of $\U{2}$.
    Then the following decomposition holds for $\alpha, \beta, \gamma, \delta \in \R$.
    \begin{equation}
        A = \e^{\iu\alpha}\mqty[\dmat[0]{\e^{-\iu \beta}, \e^{\iu \beta}}]\mqty[\cos\gamma & -\sin\gamma \\ \sin\gamma & \phantom{+}\cos\gamma]\mqty[\dmat[0]{\e^{-\iu \delta}, \e^{\iu \delta}}]
    \end{equation}
    As we can see the middle matrix is simply a 2D rotation matrix, and the other two are of a simple diagonal form.
    Lastly we have the global phase $\e^{\iu \alpha}$.
\end{example}

This is a particularly important example as the idea of decomposing unitary matrices into simpler pieces is something we will need heavily in circuit compilation tasks.
This decomposition also shows how attached to each unitary there is a phase ($\e^{\iu\alpha}$ in the above case), which in quantum computation is often irrelevant.
For that reason we also often work in the following group.
\begin{definition}
    If $A \in \U{n}$ be a unitary matrix that has the further property that
    \begin{equation}
        \det A = 1
    \end{equation}
    then $A$ is called a \emph{special unitary matrix} and we denote the collection of these matrices by $\gls{sun}$.
\end{definition}

\subsection{Quantum Gates}
A quantum gate is simply any unitary.
In this document we will mainly discuss quantum gates acting on 1 to 3 qubits since that is the range most quantum algorithms make use.
Most physical quantum computers today can perform single and two qubit gates.
\begin{table}[ht] % TODO make skinnier with text flow around?
    \centering
    % \begin{noindent}
    \begin{tabular}{cccc}
        Name           & Notation & Circuit Diagram                                                                                 & Matrix                                                 \\ \toprule
        Pauli X        & $X$      & \begin{tikzcd} \qw & \gate{X} & \qw \end{tikzcd}                                                & $\smqty[0 & 1 \\ 1 & 0]$                               \\
        Pauli Z        & $Z$      & \begin{tikzcd} \qw & \gate{Z} & \qw \end{tikzcd}                                                & $\smqty[1 & \phantom{-}0 \\ 0 & -1]$                   \\
        Hadamard       & $H$      & \begin{tikzcd} \qw & \gate{H} & \qw \end{tikzcd}                                                & $\frac{1}{\sqrt{2}}\smqty[1 & \phantom{-}1 \\ 1 & -1]$ \\
        Controlled Not & \CNOT    & \begin{tikzcd} \qw & \ctrl{1} & \qw \\ \qw & \targ{} & \qw \end{tikzcd}                         & $\smqty[1 & & & \\ & 1 & & \\ & & 0 & 1 \\ & & 1 & 0]$ \\
        Toffoli        & \CCNOT   & \begin{tikzcd} \qw & \ctrl{1} & \qw \\ \qw & \ctrl{1} & \qw \\ \qw & \targ{} & \qw \end{tikzcd} & $\smqty[1 & & & & & & & \\ & 1 & & & & & & \\ & & 1 & & & & & \\ & & & 1 & & & & \\ & & & & 1 & & & \\ & & & & & 1 & & \\ & & & & & & 0 & 1 \\ & & & & & & 1 & 0]$
    \end{tabular}
    % \end{noindent}
    \caption{Common Quantum Gates}\label{tab:commongates}
\end{table}


\begin{example}
    As we will later see, it is often important to be able to move qubits around on a physical chip.
    To do this we cannot physically move them, but rather apply some sort of combination of gates enact the swap.
    Thus we are looking for a unitary operation $\SWAP: \C^2\otimes \C^2 \to \C^2\otimes \C^2$ that acts as $\SWAP\qty[\ket{\psi}\otimes\ket{\phi}] = \ket{\phi}\otimes\ket{\psi}$.
    This is can be done using \CNOT{} gates and is shown diagrammatically.
    \begin{equation}\label{eq:cnotswap}
        \begin{quantikz}
            & \ctrl{1} & \targ{}   & \ctrl{1} & \midstick[2,brackets=none]{$\eqdef$} \qw & \swap{1} & \midstick[2,brackets=none]{$\equiv$} \qw & \gate[swap]{} & \qw \\
            & \targ{}  & \ctrl{-1} & \targ{}  & \qw                               & \targX{} & \qw                               &               & \qw
        \end{quantikz}
    \end{equation}
    Where the first equality shows us how to perform the swap with 3 \CNOT{} gates, and the last equality is an equivalence of notation.

    We can also show this using more mathematical notation if we understand the \CNOT{} map to act as $\CNOT\qty[\ket{x}\otimes\ket{y}] = \ket{x}\ket{x\oplus y}$ where $x, y \in \F$ and $\oplus$ is binary addition.
    With this we can explicitly compute the action of this circuit.
    Here we use the notation $\CNOT^a_b$ to mean a $\CNOT{}$ gate acting from qubit $a$ (the control qubit) to qubit $b$ (the target qubit).
    Then the 3 \CNOT{} gates in~\cref{eq:cnotswap} act under the following manipulations.
    \begin{align*}
        \ket{x}\otimes\ket{y} & \xrightarrow{\CNOT^1_2} \ket{x}\otimes\ket{x\oplus y}                                                     \\
                              & \xrightarrow{\CNOT^2_1} \ket{x\oplus (x\oplus y)}\otimes \ket{x\oplus y}  = \ket{y}\otimes\ket{x\oplus y} \\
                              & \xrightarrow{\CNOT^1_2} \ket{y}\otimes\ket{(x\oplus y)\oplus y} = \ket{y}\otimes \ket{x}.
    \end{align*}
    Thus exactly as desired.
\end{example}

\subsection{Quantum Circuits}

We are now ready to start putting these pieces together to build larger structures.
Since it is common that a quantum computer can perform a multitude of gates, we collect these together to form a \textbf{quantum gate set}.
These are the gates that we will be able to natively perform on our hardware.
\begin{definition}
    A \emph{quantum gate set} is a (typically finite) subset $G \subseteq \U{2^n}$. An element of $G$ is called a quantum gate.
\end{definition}
From these gates, we can construct a \textbf{quantum circuit} by applying a sequence of elements from the gate set.
\begin{definition}\label{def:circuit}
    Let $G$ be a quantum gate set, and let \gls{kleene} denote\footnote{This $^*$ operation is known as the Kleene star.} the set of finite length words over $G$ (and the empty word which we take to mean identity).
    A \emph{quantum circuit} is an element of $G^*$.
\end{definition}
Thus if our gate set $G = \qty{a, b, c}$, then some example circuits may be $aacba$, $cccbbb$, $cbbbab$, and $ab$.

Something to note here is that in this abstraction, all of our quantum gates are assumed to act on all qubits.
Hence, with a 2 qubit quantum chip and the ability to perform a Pauli $X$ gate on either qubit, then our gate is $\qty{\1\otimes \1, \1\otimes X, X\otimes \1, X\otimes X}$.\footnote{We don't always think of the identity gate $\1$ as a gate that needs to be included, but doing nothing to a qubit is no easy task, so it's important to remember to treat it just like any other gate and understand it's error rates as well.}
Sometimes this gate set is denoted $\qty{\1, X_0, X_1, X_0X_1}$, but we will try to use more explicit notation here.

Circuits are often drawn using diagram as in~\cref{fig:excircuit} where each horizontal ``wire'' represents a qubit, and boxes and other gadgets represent quantum gates.
\begin{figure}[ht]
    \centering
    \begin{quantikz}
        & \gate{U_0} & \ctrl{1} & \gate{U_1} & \ctrl{1}            & \qw           & \gate[wires=3]{U_3} & \qw \\
        & \gate{U_0} & \targ{}  & \qw        & \gate[wires=2]{U_2} & \gate[swap]{} &                     & \qw \\
        & \gate{U_0} & \qw      & \qw        &                     &               &                     & \qw
    \end{quantikz}
    \caption{Example Quantum Circuit}\label{fig:excircuit}
\end{figure}
That said, the way our theoretical model sees this circuit is more like that of~\cref{fig:abstractcircuit} where each gate acts on the entirety of the qubits.
\begin{figure}[ht]
    \centering
    \begin{quantikz}
        & \gate[wires=3][0.8cm]{A} & \gate[wires=3][0.8cm]{B} & \gate[wires=3][0.8cm]{C} & \gate[wires=3][0.8cm]{D} & \gate[wires=3][0.8cm]{E} & \gate[wires=3][0.8cm]{F} & \qw \\
        &                   &                   &                   &                   &                   &                   & \qw \\
        &                   &                   &                   &                   &                   &                   & \qw
    \end{quantikz}
    \caption{Abstract Quantum Circuit}\label{fig:abstractcircuit}
\end{figure}
In~\cref{tab:gates2circuit} we see what each one of these circuits are under the hood, and we can know that all of them are in the gate set for the above circuit.
\begin{table}[ht]
    \centering\begin{tabular}{cc}
        Gate Name & Composition                   \\ \toprule
        $A$       & $U_0 \otimes U_0 \otimes U_0$ \\
        $B$       & $\CNOT \otimes \1$            \\
        $C$       & $U_1 \otimes \1 \otimes \1$   \\
        $D$       & $\controlled{U_2}$            \\
        $E$       & $\1 \otimes \SWAP$            \\
        $F$       & $U_3$                         \\
    \end{tabular}
    \caption{Gate Compositions}\label{tab:gates2circuit}
\end{table}

We now have the machinery for circuits, and one of the important questions we need to ask is \emph{when are two circuits the same?}
Surely we can compare the circuits as strings in $G^*$, but if $G = \qty{\1, X}$, it will not tell us that $C = \1$ and $C' = XX$ are the same despite the same physical process happening.
To this end we wish to understand how the combinations of gates come together to form the entire process.
Following~\cite{formalcircuit} we define a map $\dbrackets{-}: G^* \to \U{2^n}$ which takes a quantum circuit, or sequence of gates, and multiplies them together to obtain a single unitary operator: $\dbrackets{g_1g_2\cdots g_m} = g_m\cdot g_{m-1} \cdots g_1$.\footnote{Notice here on the left we have string concatenation, and on the right matrix multiplication. Also note the fact that when doing the multiplication we reverse the order. This is an artifact of the way we draw quantum circuits from left to write, but apply gates mathematically right to left.}
Then we can say two circuits $C$ and $C'$ are effectively the same if $\dbrackets{C} = \dbrackets{C'}$.

This formalism also allows us to frame the following important question about unitary synthesis.
\begin{question}\label{qu:synthesis}
    Given a quantum gate set $G$, and unitary operator $U\in\U{2^n}$, does there exist a circuit $C\in G^*$, such that $\dbrackets{C} = U$?
\end{question}
If the answer is yes, we say a gate set $G$ \textbf{synthesizes} $U$.
This question is answered, at least in part through the Solovay-Kitaev theorem first published in~\cite{bigkitaev} with further proofs/elucidations in~\cite{nielsenchuang,solovay-kitaev,kitaev-book}.
The theorem, stated in our terminology follows.
\begin{theorem}[Solovay-Kitaev]\label{thm:solovaykitaev}
    Let $G$ be a quantum gate set such that
    \begin{itemize}
        \item $g\in\SU{2^n}$ for all $g\in G$,
        \item $g^\dagger \in G$ for all $g\in G$, and
        \item the free group $\free{G}$ is dense in $\SU{2^n}$.
    \end{itemize}
    Take $\varepsilon > 0$.
    Then, there is a constant $c > 3$, such that for any $U\in\SU{2^n}$, there exists a circuit $C\in G^*$ of length $\order{\log^c\qty(\frac{1}{\varepsilon})}$ that approximates $U$: that is $\norm{\dbrackets{C} - U} < \varepsilon$. % TODO: a lot of undefined stuff here \norm, \order (assumed, but not stated), free group
\end{theorem}
Not only does this theoretical result provide some insight into~\ref{qu:synthesis}, but it's constructive and hence provides an algorithm\footnote{This result sometimes goes under the name ``The Solovay-Kitaev Algorithm''.} to approximate arbitrary elements of $\SU{2^n}$ using gates from $\SU{2^m}$ for any $m \in \gls{intsn}$.
This result was an important and motivating because it showed that one only need a finite number of gates to perform any unitary.

With at least a partial answer to Question~\ref{qu:synthesis} we can begin to refine further questions.
If the answer to~\ref{qu:synthesis} is positive, we can then ask the following.
\begin{question}\label{qu:optimalsynthesis}
    If $G$ synthesizes $U$, and if $f: G^* \to \R$ is a cost function, can we find
    \begin{equation*}
        C_\text{min} = \argmin_{C\in G^*} \qty{f(C) : \dbrackets{C} = U}?
    \end{equation*}
\end{question}
Some examples of common cost functions are given below, and multiple can be used in the case of tie-breaking.
\begin{itemize}
    \item $f(C) = \mathtt{length}(C)$ (commonly referred to as the depth of the circuit)
    \item $f(C) = $ \# of uses of a particular gate in $C$
    \item $f(C) = \mathtt{duration}(C)$ (by this we mean the total elapsed time the circuit takes)\footnote{We have not discussed this yet, but each gate $g\in C$ takes a nonzero amount of time, during which the computation may be disturbed by outside forces.}
\end{itemize}

\subsection{Universal Gate Sets}\label{sec:universal}

We slightly danced around the idea of universality in~\cref{thm:solovaykitaev}, but we will make it clear now.
In order to harness the full power of a quantum computer, we hope it to be able to perform arbitrary unitary operations.
\begin{definition}
    A gate set $G$ on $n$ qubits is called \emph{universal} if for all $U\in\U{2^n}$ there exists a circuit $C\in G^*$ such that $\dbrackets{C} = U$.
\end{definition}
If our gate set is not universal, then we can often find ourselves in a situation where it is more efficient to simulate a given quantum algorithm than to actually run it.
\Eg{} circuits composed of gates from $\qty{\CNOT, H, S}$ are known to be efficiently simulable~\cite{gottesman-knill} despite not limiting factors typically thought to make quantum computation more powerful such as entanglement.

The question of which gate sets are universal for quantum computation is important both for our theoretical understanding of quantum computation, but also for building physical devices.
Examples that have been shown to be universal are
\begin{itemize}
    \item \CNOT{} plus $\U{2}$ as shown in~\cite{universal-cnot-u2}
    \item \CNOT{}, Hadamard, and the $\frac{\pi}{8}$-gate\footnote{The $\frac{\pi}{8}$-gate is also sometimes called $T$ and has matrix representation $\smqty[1 & 0 \\ 0 & \e^{\iu\pi/4}]$.} as shown in~\cite{universal-cnot-had-p8}
    \item \CCNOT{} (Toffoli), Hadamard, and the $\frac{\pi}{4}$-gate\footnote{The $\frac{\pi}{4}$-gate is also sometimes called $S$ and has matrix representation $\smqty[1 & 0 \\ 0 & \e^{\iu\pi/2}]$.} as shown in~\cite{bigkitaev}
    \item \CNOT{} plus any single qubit gate that does not preserve the computational basis and is not the Hadamard gate as shown in~\cite{universal-cnot-basis-change}
\end{itemize}

\section{Mathematics} % TODO cover some math for solovay-kitaev

Before moving on there are a few more bits of mathematics we need to cover.
All of our discussions in this section will assume our vector space $V$ is some complex space $\C^n$.

\subsection{Operator Norms}

\paragraph{Vector induced norms:}
Suppose our vector space $V$ has an existing norm defined on it $\norm{\cdot}: V \to \R$.
This induces a norm on the space of operators \gls{endV} as
\begin{equation}
    \norm{A}_\text{vec} \defeq \max_{v \in V}\qty{\norm{A v}: \norm{v} = 1}.
\end{equation}

\paragraph{Trace norm:}
\begin{equation}
    \norm{A}_\text{tr} \defeq \tr(\sqrt{A^\dagger A})
\end{equation}

\paragraph{Frobenius norm:}
\begin{equation}
    \norm{A}_\text{F} \defeq \sqrt{\tr(A^\dagger A)} = \qty(\sum_{i, j \in [n]}\abs{a_{ij}}^2)^{1/2} = \norm{\vectorize(A)}
\end{equation}


\subsection{Free Group}
We will not attempt a rigorous definition of the free group and instead opt for something more informal since we will not need to work with the details.
Let $S$ be a finite set, and denote by $S^{-1}$ the formal inverse of elements in $S$.
Then the free group $\free{S} \defeq (S\cup S^{-1})^*$ where the asterisk indicates the Kleene star.

As an example take $S = \qty{f, g, h}$, and hence $S^{-1} = \qty{f^{-1}, g^{-1}, h^{-1}}$.
Then the free group $\free{S}$ contains elements such as $fg^{-1}hhhh^{-1}$, $fghhhf$, and $h^{-1}gh^{-1}f^{-1}ghgg$.
Note that this may appear very similar to~\cref{def:circuit}, however we did not require our ``words'' to be over the inverses as we have here; only elements of the set itself.\footnote{This is done because in practice, being able to perform a quantum gate $U$, does not always imply we can easily perform it's inverse $U^\dagger$.} % TODO: cref not working properly with named stuff

\subsection{Dense-ness}

What does it mean for a gate set $G$ to be dense in $\SU{2^n}$?
It means that for every $U \in \SU{2^n}$, and every $\varepsilon > 0$, we have a sequence of gates $C = g_1g_2\cdots g_m$ such that $\norm{\dbrackets{C} - U} < \varepsilon$.

\subsection{Fidelity}

Suppose $\rho$ and $\sigma$ are two density operators.
Their \textbf{fidelity} is a measure of how similar the two states are and is given by
\begin{equation}
    \Fid(\rho, \sigma) \defeq \norm{\sqrt{\rho}\sqrt{\sigma}}_\text{tr} = \tr(\sqrt{\sqrt{\sigma}\rho\sqrt{\sigma}}).
\end{equation}
We can see if $\rho = \sigma$, then their fidelity is equal to 1, and otherwise this value lies between 0 and 1.
This notion can be expanded from quantum states to quantum gates to obtain a measure of how well one unitary approximates another~\cite{fidelity}.
Since $\Fid(\rho, \sigma)$ is a measure of distance between states $\rho$ and $\sigma$, and we are interested in understanding the difference of a quantum circuit and a specified target unitary, then we can calculate $\Fid(U\rho U^\dagger, \mathcal{E}(\rho))$.\footnote{Here we are using $\mathcal{E}$ to denote some quantum circuit, and $\mathcal{E}(\rho)$ it's action on $\rho$.}\todo{maybe change this notation.}
This is great, however to study a circuit more effectively, we'd like to ditch the associated state $\rho$ since we don't want to perform this analysis for every such $\rho$.
We can then define the \textbf{average gate fidelity} of a unitary $U$ and quantum circuit $\mathcal{E}$ as
\begin{equation}
    \Fid_\text{gate}(U, \mathcal{E}) \defeq \int \Fid(U\rho U^\dagger, \mathcal{E}(\rho))\dd{\rho}
\end{equation}
where the integral $\int\dd{\rho}$ is taken over all density operators.\todo{is this even right?}


