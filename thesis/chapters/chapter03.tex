%************************************************
\chapter{Quantum Hardware}\label{ch:hardware}
%************************************************

In this chapter we will give a brief overview of the necessary ideas one needs in order to understand the qubit mapping problem.

\section{Requirements}
As we saw in~\cref{sec:history} the first realization of a qubit happened in~\citeyear{firstqubit}.
A year later theoretical physicist David DiVincenzo published~\cite{divincenzo} where he outlined 5 requirements to make an effective quantum information processing device.
\begin{requirements}
    \item A scalable physical system with well characterized qubits.\label{req:scalable}
    \item The ability to initialize the state of the qubits to a simple fiducial state, such as $\ket{0}^{\otimes n}$.\label{req:initialize}
    \item Long, relevant decoherence times, much longer than the gate operation time.\label{req:Ttimes}
    \item A universal set of quantum gates.\label{req:universal}
    \item A qubit specific measurement capability.\label{req:measure}
\end{requirements}
While not an entirely revolutionary list at the time of writing, having concrete goals and checkpoints for the community to rally around/focus on meant maintainable, and trackable progress.
Again at the time of writing,~\cref{req:initialize,req:universal,req:measure} have mostly been taken care of and~\cref{req:Ttimes,req:scalable} are the areas we need the most improvement.

\section{Quantum Chips}

\begin{definition}
    A \emph{quantum chip's topology} is a graph $G = (V, E)$ with the vertices representing qubits, and edges representing connections between qubits where quantum gates of the correct arity can be applied.
\end{definition}

As an example take the the topology of IBM's \texttt{ibmq\_jakarta} shown in~\ref{fig:ibm-jakarta}.
``3'' being connected to ``1'' means that we can apply a 2-qubit unitary targeting both of those qubits, however the hardware does not support native 2-qubit gates between qubits ``2'' and ``6''.
\begin{figure}[ht]
    \centering
    \begin{tikzpicture}
        \tikzset{
            node/.style={
                    very thick,
                    circle,
                    fill=blue!20,
                    draw,
                    minimum size=.8cm,
                }
        }
        \node[node] (0) {$0$};
        \node[node] (1) [right = .8cm of 0] {$1$};
        \node[node] (2) [right = .8cm of 1] {$2$};
        \node[node] (3) [below = .8cm of 1] {$3$};
        \node[node] (5) [below = .8cm of 3] {$5$};
        \node[node] (4) [left  = .8cm of 5] {$4$};
        \node[node] (6) [right = .8cm of 5] {$6$};

        \path[draw, very thick]
        (0) edge node {} (1)
        (1) edge node {} (2)
        (1) edge node {} (3)
        (3) edge node {} (5)
        (5) edge node {} (4)
        (5) edge node {} (6);
    \end{tikzpicture}
    \caption{IBMQ Jakarta Architecture}\label{fig:ibm-jakarta}
\end{figure}

We can now define the main problem of quantum circuit compilation: that of the qubit mapping problem.\footnote{This sometimes also goes by the name of the qubit routing problem, or qubit scheduling problem although sometimes these mean slightly different things.}
If we have both a quantum circuit $C\in G^*$, and a quantum computer with network topology $G$, can the computer perform our circuit?
In order to address this question we must first talk about universal gate sets.

% In \cite{optimalSyn}

\subsection{Hardware Specifications}

\subsubsection{Relaxation and Dephasing Times}

As qubits are two-state systems, they are often implemented experimentally using some physical system (\eg{} an atom) that has a ground state, and an excited state.
Excited states often have a tendency to ``decay'' into ground states, especially so when interacting with the environment.
Hence we define the \textbf{relaxation time}\footnote{This value also goes by the following names: coherence time, amplitude damping, longitudinal coherence time, spin lattice time, and spontaneous emission time.} $T_1$ as the lifetime for the state $\ket{1}$ decaying into $\ket{0}$.
This value can be experimentally found using the following methodology.
\begin{enumerate}
    \item Prepare the state $\ket{0}$
    \item Apply a Pauli $X$ gate to obtain $\ket{1}$
    \item Wait some time $t$ (during this time the qubit may decay into $\ket{0}$)
    \item Measure the qubit
\end{enumerate}
Each time we measure the qubit in the ground state we record the amount of time $t$ we waited.
This process is then modelled with an exponential decay of the form $\e^{-t/T_1}$.

The second important factor we need to understand is the \textbf{dephasing time}\footnote{Again, this value also goes by the following names: phase coherence time, phase damping, spin-spin relaxation time, transverse coherence time, and elastic scattering time.}
This time, instead of watching for the the bit flip from $\ket{1}$ to $\ket{0}$ we will watch for a phase flip from $\ket{+}$ to $\ket{-}$ via the following procedure.
\begin{enumerate}
    \item Prepare the state $\ket{0}$
    \item Apply a Hadamard $H$ gate to obtain $\ket{+}$
    \item Wait some time $t$ (during this time a phase might appear on either qubit)
    \item Apply another Hadamard $H$
    \item Measure the qubit
\end{enumerate}
Again, this experiment models an exponential decay with lifetime which we denote $T_2$.
This decoherence time is a measure of how quickly a superposition ($\ket{+}$) will decay into a classical mixture.
Since $T_1$ is a measure of how robust the qubit is against bit flips, and $T_2$ is a measure of how robust the qubit is against becoming probabilistic, these two quantities are very important track the progress of quantum computers.


\section{Errors}
