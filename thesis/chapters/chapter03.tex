%************************************************
\chapter{Quantum Hardware}\label{ch:hardware}
%************************************************

In this chapter we will give a brief overview of the necessary ideas one needs in order to understand the qubit mapping problem.

\section{T1}
\section{T2}
\section{CLOPS}


\section{Quantum Compilers}
We can now return to the issue of compilers.
It should now be clear that the level of abstraction we work at when designing quantum algorithms is much higher than the capabilities of our current, and likely future hardware.
Hence quantum compilers are needed for two steps.
\begin{figure}[h]
    \centering
    \tikzset{
        frame/.style={
                draw,
                text width=6em,
                text centered,
                minimum height=4em,
                drop shadow,
                fill=orange!40,
                rounded corners,
            },
    }
    \begin{tikzpicture}[font=\sffamily, very thick, node distance=4cm]
        \node[align=center] (qa) {Quantum \\ Algorithm};
        % \node[frame, right of=qa] (compiler) {Compiler};
        \node[frame, right of=qa] (ir) {};%Intermediate \\ Representation};
        \node[align=center, above right=.5cm and 1cm of ir] (ha) {Hardware A};
        \node[align=center, below=.5cm of ha] (hb) {Hardware B};
        \node[align=center, below=.1cm of hb] (hdots) {\vdots};

        % \node[align=center, right of=compiler] (ml) {Hardware A};

        \draw[-stealth] (qa) -- (ir);
        \draw[-stealth] (ir) -- (ha);
        \draw[-stealth] (ir) -- (hb);
        % \draw[-stealth] (ir) -- (hdots);

        \node[very thick, draw=orange!20, fit=(ir), inner sep=3mm, label=above:{Quantum Compiler}, rounded corners](qcompiler) {};
    \end{tikzpicture}
    \caption{Action of Quantum Compiler}\label{fig:quantumcompiler}
\end{figure}
