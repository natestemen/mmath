%************************************************
\chapter{Quantum Hardware}\label{ch:hardware}
%************************************************

In this chapter we will give a brief overview of the necessary ideas one needs in order to understand the qubit mapping problem.

\section{Quantum Chips}

\begin{definition}
    A \emph{quantum chip's topology} is a graph $G = (V, E)$ with the vertices representing qubits, and edges representing connections between qubits where quantum gates of the correct arity can be applied.
\end{definition}

As an example take the the topology of IBM's \texttt{ibmq\_jakarta} shown in~\ref{fig:ibm-jakarta}.
``3'' being connected to ``1'' means that we can apply a 2-qubit unitary targeting both of those qubits, however the hardware does not support native 2-qubit gates between qubits ``2'' and ``6''.
\begin{figure}[ht]
    \centering
    \begin{tikzpicture}
        \tikzset{
            node/.style={
                    very thick,
                    circle,
                    fill=blue!20,
                    draw,
                    minimum size=.8cm,
                }
        }
        \node[node] (0) {$0$};
        \node[node] (1) [right = .8cm of 0] {$1$};
        \node[node] (2) [right = .8cm of 1] {$2$};
        \node[node] (3) [below = .8cm of 1] {$3$};
        \node[node] (5) [below = .8cm of 3] {$5$};
        \node[node] (4) [left  = .8cm of 5] {$4$};
        \node[node] (6) [right = .8cm of 5] {$6$};

        \path[draw, very thick]
        (0) edge node {} (1)
        (1) edge node {} (2)
        (1) edge node {} (3)
        (3) edge node {} (5)
        (5) edge node {} (4)
        (5) edge node {} (6);
    \end{tikzpicture}
    \caption{IBMQ Jakarta Architecture}\label{fig:ibm-jakarta}
\end{figure}

We can now define the main problem of quantum circuit compilation: that of the qubit mapping problem.\footnote{This sometimes also goes by the name of the qubit routing problem, or qubit scheduling problem although sometimes these mean slightly different things.}
If we have both a quantum circuit $C\in G^*$, and a quantum computer with network topology $G$, can the computer perform our circuit?
In order to address this question we must first talk about universal gate sets.

% In \cite{optimalSyn}
\section{T1}
\section{T2}
\section{CLOPS}


\section{Quantum Compilers}
We can now return to the issue of compilers.
It should now be clear that the level of abstraction we work at when designing quantum algorithms is much higher than the capabilities of our current, and likely future hardware.
Hence quantum compilers are needed for two steps.
\begin{figure}[h]
    \centering
    \tikzset{
        frame/.style={
                draw,
                text width=6em,
                text centered,
                minimum height=4em,
                drop shadow,
                fill=orange!40,
                rounded corners,
            },
    }
    \begin{tikzpicture}[font=\sffamily, very thick, node distance=4cm]
        \node[align=center] (qa) {Quantum \\ Algorithm};
        % \node[frame, right of=qa] (compiler) {Compiler};
        \node[frame, right of=qa] (ir) {};%Intermediate \\ Representation};
        \node[align=center, above right=.5cm and 1cm of ir] (ha) {Hardware A};
        \node[align=center, below=.5cm of ha] (hb) {Hardware B};
        \node[align=center, below=.1cm of hb] (hdots) {\vdots};

        % \node[align=center, right of=compiler] (ml) {Hardware A};

        \draw[-stealth] (qa) -- (ir);
        \draw[-stealth] (ir) -- (ha);
        \draw[-stealth] (ir) -- (hb);
        % \draw[-stealth] (ir) -- (hdots);

        \node[very thick, draw=orange!20, fit=(ir), inner sep=3mm, label=above:{Quantum Compiler}, rounded corners](qcompiler) {};
    \end{tikzpicture}
    \caption{Action of Quantum Compiler}\label{fig:quantumcompiler}
\end{figure}
