% ***********************************************
\chapter{Quantum Compilers}\label{ch:qcomps}
% ***********************************************

We can now return to the issue of compilers.
It should now be clear that the level of abstraction we work at when designing quantum algorithms is much higher than the capabilities of our current, and likely future hardware.
Hence quantum compilers are needed for two steps.
\begin{figure}[h]
    \centering
    \tikzset{
        frame/.style={
                draw,
                text width=6em,
                text centered,
                minimum height=4em,
                drop shadow,
                fill=orange!40,
                rounded corners,
            },
    }
    \begin{tikzpicture}[font=\sffamily, very thick, node distance=4cm]
        \node[align=center] (qa) {Quantum \\ Algorithm};
        % \node[frame, right of=qa] (compiler) {Compiler};
        \node[frame, right of=qa] (ir) {};%Intermediate \\ Representation};
        \node[align=center, above right=.5cm and 1cm of ir] (ha) {Hardware A};
        \node[align=center, below=.5cm of ha] (hb) {Hardware B};
        \node[align=center, below=.1cm of hb] (hdots) {\vdots};

        % \node[align=center, right of=compiler] (ml) {Hardware A};

        \draw[-stealth] (qa) -- (ir);
        \draw[-stealth] (ir) -- (ha);
        \draw[-stealth] (ir) -- (hb);
        % \draw[-stealth] (ir) -- (hdots);

        \node[very thick, draw=orange!20, fit=(ir), inner sep=3mm, label=above:{Quantum Compiler}, rounded corners](qcompiler) {};
    \end{tikzpicture}
    \caption{Action of Quantum Compiler}\label{fig:quantumcompiler}
\end{figure}

\dots Thus when compiling circuits, we need to minimize the number of \SWAP{} gates we must add since we saw in~\cref{eq:cnotswap} that each \SWAP{} takes 3 \CNOT{} gates.

\paragraph{Compiling the Toffoli Gate}
Since most hardware are not capable of 3-qubit operations we must decompose the Toffoli, or \CCNOT{} gate into something more manageable.
This is typically done using \CNOT{}'s, Hadamard's ($H$), and $\pi/8$ ($T$) gates~\cite{nielsenchuang}.
\begin{equation}
    \begin{quantikz}[column sep=.25cm]
        & \ctrl{1} & \midstick[3,brackets=none]{$=$} \qw & \qw      & \qw      & \qw              & \ctrl{2} & \qw      & \qw      & \qw              & \ctrl{2} & \qw      & \ctrl{1} & \gate{T}         & \ctrl{1} & \qw \\
        & \ctrl{1} & \qw                                 & \qw      & \ctrl{1} & \qw              & \qw      & \qw      & \ctrl{1} & \qw              & \qw      & \gate{T} & \targ{}  & \gate{T^\dagger} & \targ{}  & \qw \\
        & \targ{}  & \qw                                 & \gate{H} & \targ{}  & \gate{T^\dagger} & \targ{}  & \gate{T} & \targ{}  & \gate{T^\dagger} & \targ{}  & \gate{T} & \gate{H} & \qw              & \qw      & \qw
    \end{quantikz}
\end{equation}
This is an important decomposition as the \CCNOT{} gate appears in the modular exponentiation problem which is a core part of Shor's algorithm~\cite{shor-encryption}.
Hence if there are smaller decompositions than shown above that would be ideal as \emph{one} \CCNOT{} gate becomes 14!
\citeauthor{universal-cnot-u2} show a more compact decomposition of \CCNOT{} using only 3 \CNOT{} gates if the phase of one of the qubits is allowed to change~\cite{universal-cnot-u2}.
Let $G = R_Y(\frac{\pi}{4})$ in the following circuit. % following https://arxiv.org/pdf/quant-ph/9705009.pdf rather than original paper
\begin{equation}
    \begin{quantikz}%[column sep=.25cm]
        & \ctrl{1} & \midstick[3,brackets=none]{$\approx$} \qw & \qw                        & \qw      & \qw                        & \ctrl{2} & \qw                       & \qw      & \qw                       & \qw \\
        & \ctrl{1} & \qw                                       & \qw                        & \ctrl{1} & \qw                        & \qw      & \qw                       & \ctrl{1} & \qw                       & \qw \\
        & \targ{}  & \qw                                       & \gate{G^\dagger} & \targ{}  & \gate{G^\dagger} & \targ{}  & \gate{G} & \targ{}  & \gate{G} & \qw
    \end{quantikz}
\end{equation}
However the question ended in~\citeyear{toff3cnot} when it was shown that a true equality preserving decomposition of the \CCNOT{} gate requires a minimum of 6 \CNOT{} gates~\cite{toff3cnot}.\footnote{This result shows that a minimum of 6 \CNOT{} gates must be used, \textbf{if} they are being used. Other decompositions not using \CNOT{} gates might still be more compact.}

\section{Strong and Weak Compilation}


One of the benefits of the modular compiler structure seen in~\cref{fig:compends} is that once the optimizer is made, we can write a backend to go to real hardware, \textbf{and} write another backend to send the code to classical hardware.
This in effect provides an optimized quantum simulator.

\section{Compiling on a ring}

\section{Methods}

It's important to note there that quantum circuit compilers exist and in many of the references that follow the authors' proposed compilation methods are benchmarked against the most prevalent compiler Qiskit~\cite{qiskit}.

\paragraph{QAOA}
Compilers have also been built to attack specific problems such as the \ac{QAOA}~\cite{qaoa} where particular gates can be swapped horizontally in circuit diagrams due to their commutative nature.
Focusing in this particular problem, the authors in~\cite{qaoa-compiler} have been able to reduce the gate count by 23\% and circuit depth by 53\% on average.
In the future we might hope to build these problem specific compilers into a more general purpose one that can diagnose and understand when to use problem specific compilers on demand.

\paragraph{VQE}
Another hybrid quantum-classical algorithm that has seem much attention is that of the \ac{VQE}~\cite{vqe,vqe2}.
This algorithm is used in quantum chemistry to calculate the ground state of a molecular Hamiltonian using a parametrized quantum circuit as a cost function, and the classical compute nodes as an optimizer.
\Eg{} let $\boldsymbol{\theta} \in \R^n$ be a vector of numbers that our circuit $U$ depends on, \ie{} $U: \R^n \to \U{2^m}$ for some number of qubits $m$.
\begin{figure}[ht] % TODO label positioning fucked
    \centering
    \includestandalone[width=0.8\textwidth]{tikz/vqe}
    \caption{\acs{VQE} Schematic}\label{fig:vqe}
\end{figure}
A compiler specific to this problem has been created, and generalized to further quantum-classical algorithms in~\cite{vqe-compiler} and the authors have leveraged much of the existing infrastructure brought forth by the LLVM project discussed in~\cref{sec:llvm}.
This allows the classical optimizations to be handled by the robust LLVM system, while using new circuit compilation techniques on the variational circuit.


\section{Quantum Stack}
