%************************************************
\chapter{Quantum Hardware}\label{chap:hardware}
%************************************************

The goal of this chapter is twofold.
First, we introduce the most common constraints seen in modern quantum hardware, as well as other common tools used to measure the efficacy of a given quantum computer.
Second we will introduce the mathematical formalism needed in order to formulate the problems related to quantum circuit compilation we will see in~\cref{chap:circuit-compilers}.
We will not, however, attempt to give an introduction to the physical implementations of quantum hardware and instead refer the reader to~\cite[Chapter 7]{nielsenchuang} for a more comprehensive introduction.

\section{Requirements}

In~\citeyear{divincenzo} David DiVincenzo proposed 5 requirements as being necessary to make an effective quantum information processing device~\cite{divincenzo}.
His proposed requirement are summarized here.
\begin{requirements}
    \item A scalable physical system with well-characterized qubits.\label{req:scalable}
    \item The ability to initialize the state of the qubits to a simple fiducial state, such as $\ket{0}^{\!\otimes n}$.\label{req:initialize}
    \item Long decoherence times, much longer than the gate operation time.\label{req:Ttimes}
    \item A universal set of quantum gates.\label{req:universal}
    \item A qubit measurement capability.\label{req:measure}
\end{requirements}
While all of these requirements are still under active research,~\cref{req:initialize,req:universal,req:measure} are completed for \ac{NISQ} devices, while~\cref{req:Ttimes,req:scalable} keep us in the \ac{NISQ}-era.
However, even if all of these problems were solved completely, there are still many things that can go wrong.
Just because your qubits scale doesn't mean you have enough for a specific algorithm.
Just because there is a long decoherence time doesn't mean a qubit can't error in some other way.
Just because you have a universal gate set doesn't mean you know how to efficiently decompose a gate from the algorithm you are trying to run.

As you can see, these five requirements provide us with the backbone upon which we can build further, but do not guarantee optimal quantum computations.
It is some of these secondary questions we wish to understand more deeply to make quantum computers more useful once the bedrock has been established.

\section{Quantum Chips}

Intuitively a quantum chip is a collection of qubits along with the capability to perform operations on subsets of the qubits.
This can be formalized using a graph structure as follows. % TODO better lead in

\begin{definition}\label{def:topology}
    A \emph{quantum connectivity graph} is an undirected\footnote{There are some hardware which are better modeled by a directed graph, and we will see an example in~\cref{sec:ringcomp}, but for most cases undirected is simpler and provides the intuition.} graph $H = (V, E)$\footnote{$V$ is a finite set which we refer to as vertices, and $E$ is a collection of pairs of vertices, \ie{} $E \subseteq V \times V$.} such that $(v, v) \in E$ for all $v \in V$.
\end{definition}
What we call the connectivity graph is sometimes referred to as a (network) topology in other resources.
We ensure the connectivity graph has all self loops as edges are the basis for performing quantum gates and all modern hardware has the capability of performing single qubit gates.
Once the connectivity graph has been established, we can consider which gate-sets are allowable by ensuring only qubits which are connected via an edge are acted on in a nontrivial manner.
That is, if two nodes are not connected via an edge, there should be no entangling gates operating on them. % TODO define entangling gate
\begin{definition}
    Let $H = (V, E)$ be a quantum connectivity graph.
    A gate set $G$ is said to be \emph{amenable} to $H$ if
    \begin{itemize}
        \item $G$ acts on $\abs{V}$ qubits, and
        \item $\qubits(g) \in E$ for all gates $g \in G$.\todo{this only works if gate set is defined in a particular way that doesn't allow things like $X \otimes X$ in gate set.}
    \end{itemize}
\end{definition}
We can now combine the connectivity graph and an amenable gate set to form the model of quantum hardware.
\begin{definition}
    A \emph{quantum chip} $T = (H, G)$ is a quantum connectivity graph $H = (V, E)$ together with an amenable gate set $G$.
    The gates $g \in G$ are often called \emph{native} to $T$.
\end{definition}

While this formalism does have the drawback of restricting our gate sets to single and two qubit gates, this model applies to the majority of hardware today.
Multi-qubit gates can be allowed using the notion of a hypergraph in~\cref{def:topology}, but doing so introduces complexity without a clear advantage.
This slightly simplified notion still encompasses universal quantum computation as three (and higher) qubit gates are not needed as we saw in~\cref{sec:universal}.

As an example the connectivity graph of IBM's 7-qubit quantum computer \texttt{ibmq\_jakarta} shown in~\cref{fig:ibm-jakarta} and the gate set is as follows~\cite{ibmq}.
\begin{equation} % TODO this doesn't align with how we've defined gate sets in chap 2
    \qty{\CNOT, \1, R_Z, S_X, X}
\end{equation}
Vertex ``3'' being connected to ``1'' means that we can apply a 2-qubit unitary targeting both of those qubits, however the hardware does not support 2-qubit gates between qubits ``2'' and ``6'' natively.
\begin{figure}[ht]
    \centering
    \includestandalone[width=0.4\textwidth]{tikz/jakarta}
    \caption{IBMQ Jakarta Architecture}\label{fig:ibm-jakarta}
\end{figure}

\begin{definition} % TODO def needs to be moved up
    Let $G$ be a quantum gate set acting on $n$ qubits.
    Define a function $\qubits: G \to \mathcal{P}([n])$ which returns a set containing the index of the qubits each gate acts on nontrivially.
\end{definition}
As an example if $G = \qty{\1 \otimes \1, \1 \otimes X, X \otimes \1, X \otimes X}$, then $\qubits(\1 \otimes \1) = \emptyset$, $\qubits(\1 \otimes X) = \qty{1}$, and $\qubits(X \otimes X) = \qty{0, 1}$.

\begin{definition}
    Let $T = (G, H)$ be a quantum chip with graph $H = (V, E)$, and $C \in G^*$ be a circuit.
    The quantum chip $T$ can run $C$ if for all $g \in C$ we have $\qubits(g) \in E$.
    In this case we say that $C$ is \emph{executable} on $T$.
\end{definition}

We can now define the main problem of quantum circuit compilation: that of the qubit mapping problem.\footnote{This sometimes also goes by the name of the qubit routing problem, or qubit scheduling problem although sometimes these mean slightly different things.}
\begin{question}\label{qu:qubitmapping}
    Let $C \in A^*$ be a circuit over quantum gate set $A$, and $T = (B, H)$ a quantum chip.
    Is there a $T$-executable circuit $C' \in B^*$ such that $\implement{C'} = \implement{C}$?
\end{question}
In the case when the number of qubits required in $C$ is greater than the number of vertices in the connectivity graph, the answer is no.\footnote{There are however specific cases when this \emph{is} possible. If the algorithm only requires $n < \abs{V}$ qubits to be entangled at once, there are clever scheduling tactics one can employ to implement such an algorithm. There are also ``quantum autoencoders'' which attempt to implement compressed versions of circuits on smaller numbers of qubits~\cite{autoencode}.}
On the other hand when the number of qubits required for $C$ is fewer than the number we have access to $\abs{V}$, then the answer is yes, provided
\begin{enumerate}
    \item $B$ is a universal gate set, and
    \item $(V, E)$ is connected.\footnote{That is for any two vertices, there is a path between them.}
\end{enumerate}
Just as the unitary synthesis problem (\cref{qu:synthesis}) was turned into an optimization problem by the use of cost functions (\cref{qu:optimalsynthesis}), we can ask for the optimal version of~\cref{qu:qubitmapping}.
\begin{question}
    Let $T = (B, H)$ be a quantum chip and $C \in B^*$ be a $T$-executable circuit that implements $C' \in A^*$ (\ie{} $\implement{C} = \implement{C'}$).
    Let $f: B^* \to \R$ be a cost function.
    Can we find
    \begin{equation}
        C_\text{min} = \argmin_{C \in B^*}\qty{f(C) : \implement{C} = \implement{C'} \quad\text{and}\quad C\text{ is $T$-executable}}?
    \end{equation}
\end{question}

Despite formulating the key problem we'd like to understand in this document, there are many contributing factors that effect solutions to this problem
For that reason we need to not just understand a theoretical model of quantum hardware, but some of the implementation details as well.

\section{Hardware Specifications}

Here we will briefly cover the most important topics discussed in quantum circuit compilation when it comes to optimizations on \ac{NISQ}-era hardware.

\subsubsection{Relaxation and Dephasing Times}\label{sec:Ttimes}

As qubits are two-state systems, they are often implemented experimentally using some physical system (\eg{} an atom) that has a ground state, and an excited state.
Excited states often have a tendency to ``decay'' into ground states, especially so when interacting with the environment.
Hence we define the \textbf{relaxation time}\footnote{This value also goes by the following names: coherence time, amplitude damping, longitudinal coherence time, spin lattice time, and spontaneous emission time.} $T_1$ as the lifetime for the state $\ket{1}$ decaying into $\ket{0}$.
This value can be experimentally found using the following methodology.
\begin{enumerate}
    \item Prepare the state $\ket{0}$
    \item Apply a Pauli $X$ gate to obtain $\ket{1}$
    \item Wait some time $t$ (during this time the qubit may decay into $\ket{0}$)
    \item Measure the qubit
\end{enumerate}
Each time we measure the qubit in the ground state we record the amount of time $t$ we waited.
This process is then modelled with an exponential decay of the form $\e^{-t/T_1}$.

The second important factor we need to understand is the \textbf{dephasing time}\footnote{Again, this value also goes by the following names: phase coherence time, phase damping, spin-spin relaxation time, transverse coherence time, and elastic scattering time.}
This time, instead of watching for the the bit flip from $\ket{1}$ to $\ket{0}$ we will watch for a phase flip from $\ket{+}$ to $\ket{-}$ via the following procedure.
\begin{enumerate}
    \item Prepare the state $\ket{0}$
    \item Apply a Hadamard $H$ gate to obtain $\ket{+}$
    \item Wait some time $t$ (during this time a phase might appear on either qubit)
    \item Apply another Hadamard $H$
    \item Measure the qubit
\end{enumerate}
Again, this experiment is modeled by an exponential decay with lifetime which we denote $T_2$.
This decoherence time is a measure of how quickly a superposition ($\ket{+}$) will decay into a classical mixture.
In both the definition of $T_1$ and $T_2$ step 3 requires the experimenter to ``wait'', meaning apply identity gates until time $t$.
That said while the waiting occurs, if \emph{other} qubits are acted upon, this may change the experimental results due to crosstalk (\cref{sec:errors}).
Since $T_1$ is a measure of how robust the qubit is against bit flips, and $T_2$ is a measure of how robust the qubit is against becoming probabilistic, these two quantities are important metrics to track the progress of quantum computers.

\subsubsection{Quantum Volume}

While transistor count has long served as an effective single number metric for the power of classical computers, qubit count does not have the same descriptive power due to quantum computers' difference in connectivity, and error rates.
% Researchers have found that despite isolating qubits to the best of their ability, when sending instructions to a qubit (\eg{} a gate), nearby qubits can ``hear'' such a message and be affected by it~\cite{crosstalk,crosstalk2}.
% With the introduction of errors that are not solely from a single qubit or gate, it makes sense to have a measure of how effective the quantum computer is as a whole.
In attempt to devise a single number metric effective in quantifying a quantum computers capabilities \citeauthor{qvolume} introduced the notion of \textbf{quantum volume} which takes into account the number of qubits, connectivity, gate and measurement errors, and crosstalk.
While understanding the full method to measure a quantum computer's quantum volume is beyond the scope of this document, the process consists of applying randomized circuits shown in~\cref{fig:qvolume} where $\pi$ is a permutation of the qubits, and $\PU{4}$ denotes a random two-qubit gate.
After the gates are applied a measurement is performed and the resulting bit-string is stored, and the process is repeated many times.
A statistical analysis is then run to compare the computers performance with an ideal implementation of random circuits of this form.
The largest quantum volume achieved to date is 1024 and was done by Honeywell's System Model H1; a 10-qubit trapped-ion computer~\cite{honeywell}.

\begin{figure}[ht]
    \centering
    \begin{quantikz}
        & \gate[wires=4]{\pi}\gategroup[4,steps=2,style={dashed,rounded corners,fill=blue!20, inner xsep=2pt}, background]{Round 1} & \gate[wires=2]{\PU{4}} & \gate[wires=4]{\pi}\gategroup[4,steps=2,style={dashed,rounded corners,fill=blue!20, inner xsep=2pt}, background]{Round 2} & \gate[wires=2]{\PU{4}} & \ \ldots\ \qw & \gate[wires=4]{\pi}\gategroup[4,steps=2,style={dashed,rounded corners,fill=blue!20, inner xsep=2pt}, background]{Round $d$} & \gate[wires=2]{\PU{4}} & \qw \\
        &                     &                        &                     &                        & \ \ldots\ \qw &                     &                        & \qw \\
        &                     & \gate[wires=2]{\PU{4}} &                     & \gate[wires=2]{\PU{4}} & \ \ldots\ \qw &                     & \gate[wires=2]{\PU{4}} & \qw \\
        &                     &                        &                     &                        & \ \ldots\ \qw &                     &                        & \qw
    \end{quantikz}
    \caption{Quantum Volume Protocol}\label{fig:qvolume}
\end{figure}

\subsubsection{Gate Duration}

As qubits are finnicky beasts that don't want to retain their quantum-ness, how quickly we can perform gates is a very important measure and one tracked across many quantum computers.
This measure usually comes under the guise of \ac{CLOPS} first introduced in~\cite{clops}. % TODO define circuit layer

\section{Errors}\label{sec:errors}

Errors are ubiquitous in quantum computing and for the near future there is almost certainly no getting around them.
Not only are errors abundant, but they can vary across the chip, and they can very in type.
The first error that is often encountered is that of \textbf{gate errors}.
Some examples of gate errors might be
\begin{itemize}
    \item performing $R_X(\theta + \varepsilon)$ when you intended to do $R_X(\theta)$, or
    \item performing $H + \varepsilon X$ when you intended to apply $H$.
\end{itemize}
This first type of error is sometimes mitigated experimentally if $\varepsilon$ is either fixed, or coupled in some way to $\theta$.
However it may be the case that a more complex coupling is taking place dependent on the surrounding state of the qubit that the gate is acting on.
Since we represent a quantum chip as a graph, one way to quantify errors is to attach a number to each node and edge.
The node error rate represents the computers error rate on performing a single qubit unitary, and the edge represents the computers error rate for performing a 2-qubit unitary.

The next type of error that can be introduced into a quantum computation is through \textbf{\ac{SPAM}} errors. % TODO it's bolding the acro :(
These---as you might have guessed---are introduced during state preparation and measurement.
Despite these being one of the largest sources of errors on modern quantum hardware, quantum circuit compilation cannot aid in mitigating these errors.

Finally, the last major source of noise that is seen in quantum computers is that of \textbf{crosstalk}.
Crosstalk corrupts information in our system when multiple gates are performed simultaneously.
This is unfortunate as parallelizing computation drastically improves the running time and keeps the total computation time below decoherence times discussed in~\cref{sec:Ttimes}.
These errors arise as qubits are not perfectly isolated from each other and hence can interact especially when control pulses (\ie{} the gate implementations) bleed into nearby qubits.
