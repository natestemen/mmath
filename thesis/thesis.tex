% ******************************************************
% A Classic Thesis Style
% An Homage to The Elements of Typographic Style
%
% Copyright (C) 2018 André Miede and Ivo Pletikosić
% ******************************************************
\RequirePackage{silence}
    \WarningFilter{scrreprt}{Usage of this version of package `titlesec'}
    \WarningFilter{scrreprt}{Usage of package `tocloft' together}
    \WarningFilter{titlesec}{Non standard sectioning command}
\documentclass[
    twoside,
    openright,
    titlepage,
    numbers=noenddot,
    headinclude,
    footinclude,
    cleardoublepage=empty,
    abstract=on,
    BCOR=5mm,
    paper=a4,
    fontsize=11pt
]{scrreprt}

% ******************************************************
% Note: Make all your adjustments in here
% ******************************************************
\PassOptionsToPackage{utf8}{inputenc}
\usepackage{inputenc}

\PassOptionsToPackage{T1}{fontenc}
\usepackage{fontenc}


% ********************************************************************
% 1. Configure classicthesis for your needs here, e.g., remove "drafting" below
% in order to deactivate the time-stamp on the pages
% (see ClassicThesis.pdf for more information):
% ********************************************************************
\PassOptionsToPackage{
    drafting=true,    % print version information on the bottom of the pages
    tocaligned=false, % the left column of the toc will be aligned (no indentation)
    dottedtoc=false,  % page numbers in ToC flushed right
    eulerchapternumbers=true, % use AMS Euler for chapter font (otherwise Palatino)
    linedheaders=false,       % chaper headers will have line above and beneath
    floatperchapter=true,     % numbering per chapter for all floats (i.e., Figure 1.1)
    eulermath=false,  % use awesome Euler fonts for mathematical formulae (only with pdfLaTeX)
    beramono=true,    % toggle a nice monospaced font (w/ bold)
    palatino=true,    % deactivate standard font for loading another one, see the last section at the end of this file for suggestions
    style=classicthesis % classicthesis, arsclassica
}{classicthesis}


% ********************************************************************
% 2. Personal data and user ad-hoc commands (insert your own data here)
% ********************************************************************
\newcommand{\myTitle}{Quantum Circuit Compilation From The Ground Up\xspace}
\newcommand{\mySubtitle}{\xspace}
\newcommand{\myDegree}{Master of Mathematics (MMath)\xspace}
\newcommand{\myName}{Nathaniel Stemen\xspace}
% \newcommand{\myProf}{Someone else TODO\xspace}
\newcommand{\mySupervisor}{Joel Wallman\xspace}
\newcommand{\myFaculty}{Faculty of Mathematics\xspace}
\newcommand{\myDepartment}{Department of Applied Mathematics\xspace}
\newcommand{\myUni}{University of Waterloo\xspace}
\newcommand{\myInst}{Institute for Quantum Computing\xspace}
\newcommand{\myLocation}{Seattle, WA\xspace}
\newcommand{\myTime}{April 2022\xspace}
\newcommand{\myTimeFrame}{September 2020---April 2022}
\newcommand{\myVersion}{\classicthesis}

% ********************************************************************
% 3. Loading some handy packages
% ********************************************************************

% ********************************************************************
% Packages with options that might require adjustments
% ********************************************************************
\PassOptionsToPackage{american}{babel}
\usepackage{babel}

\usepackage{csquotes}

\PassOptionsToPackage{
    style=alphabetic,
    maxnames=5,
    backref=true
}{biblatex}
\usepackage{biblatex}

\PassOptionsToPackage{fleqn}{amsmath} % float equations towards left side
\usepackage{amsmath}
\usepackage{amssymb}
\newtheorem{theorem}{Theorem}[section]
\newtheorem{definition}[theorem]{Definition}
\newtheorem{example}[theorem]{Example}
\newtheorem{question}[theorem]{Question}

% ********************************************************************
% General useful packages
% ********************************************************************
\usepackage{graphicx}
\usepackage{scrhack} % fix warnings when using KOMA with listings package
\usepackage{xspace} % to get the spacing after macros right
\PassOptionsToPackage{printonlyused,smaller}{acronym}
\usepackage{acronym} % TODO change to acro package
\usepackage{physics}
\usepackage{mathtools}
\usepackage{stmaryrd}
\DeclarePairedDelimiter\implement{\llbracket}{\rrbracket}
\usepackage{xfrac}


% ********************************************************************
% 4. Setup floats: tables, (sub)figures, and captions
% ********************************************************************
\usepackage{tabularx} % better tables: TODO change to booktabs
\setlength{\extrarowheight}{3pt} % increase table row height
\newcommand{\tableheadline}[1]{\multicolumn{1}{l}{\spacedlowsmallcaps{#1}}}
\newcommand{\myfloatalign}{\centering} % to be used with each float for alignment
\usepackage{subfig}


% ********************************************************************
% 5. Setup code listings
% ********************************************************************
\usepackage{listings}
%\lstset{emph={trueIndex,root},emphstyle=\color{BlueViolet}}%\underbar} % for special keywords
\lstset{language=[LaTeX]Tex,%C++,
    morekeywords={PassOptionsToPackage,selectlanguage},
    keywordstyle=\color{RoyalBlue},%\bfseries,
    basicstyle=\small\ttfamily,
    %identifierstyle=\color{NavyBlue},
    commentstyle=\color{Green}\ttfamily,
    stringstyle=\rmfamily,
    numbers=none,%left,%
    numberstyle=\scriptsize,%\tiny
    stepnumber=5,
    numbersep=8pt,
    showstringspaces=false,
    breaklines=true,
    %frameround=ftff,
    %frame=single,
    belowcaptionskip=.75\baselineskip
    %frame=L
}


% ********************************************************************
% 6. Last calls before the bar closes
% ********************************************************************
\usepackage{classicthesis}


% ********************************************************************
% Fine-tune hyperreferences (hyperref should be called last)
% ********************************************************************
\hypersetup{%
    % COLOR ***********
    colorlinks=true,
    linktocpage=true,
    urlcolor=CTurl, linkcolor=CTlink, citecolor=CTcitation,
    % urlcolor=Black, linkcolor=Black, citecolor=Black, % for printing
    % IDK TBH *********
    breaklinks=true,
    pageanchor=true,
    bookmarksnumbered=true, bookmarksopen=true, bookmarksopenlevel=1,
    % PDF META ********
    pdftitle={\myTitle},
    pdfauthor={\textcopyright\ \myName, \myUni},
    pdfsubject={Quantum Circuit Compilation},
    pdfkeywords={quantum information, quantum computation, quantum circuits, compilers},
    pdfcreator={pdfLaTeX},
    pdfproducer={LaTeX with hyperref and classicthesis},
}


% ********************************************************************
% Setup autoreferences (hyperref and babel)
% ********************************************************************
% There are some issues regarding autorefnames
% http://www.tex.ac.uk/cgi-bin/texfaq2html?label=latexwords
% you have to redefine the macros for the
% language you use, e.g., american, ngerman
% (as chosen when loading babel/AtBeginDocument)
% ********************************************************************
\makeatletter
\@ifpackageloaded{babel}%
{%
    \addto\extrasamerican{%
        \renewcommand*{\figureautorefname}{Figure}%
        \renewcommand*{\tableautorefname}{Table}%
        \renewcommand*{\partautorefname}{Part}%
        \renewcommand*{\chapterautorefname}{Chapter}%
        \renewcommand*{\sectionautorefname}{Section}%
        \renewcommand*{\subsectionautorefname}{Section}%
        \renewcommand*{\subsubsectionautorefname}{Section}%
    }%
    % Fix to getting autorefs for subfigures right (thanks to Belinda Vogt for changing the definition)
    \providecommand{\subfigureautorefname}{\figureautorefname}%
}{\relax}
\makeatother


\newcommand{\R}{\mathbb{R}}
\newcommand{\C}{\mathbb{C}}
\newcommand{\F}{\mathbb{F}}
\newcommand{\1}{\mathbb{1}}
\newcommand{\iu}{\mkern1mu\mathrm{i}\mkern1mu}
\newcommand{\e}{\mathrm{e}}
\newcommand{\mats}[2]{\mathcal{M}_{#1}(#2)}
\newcommand{\U}[1]{\mathsf{U} (#1)}
\newcommand{\PU}[1]{\mathsf{PU} (#1)}

\newcommand{\CNOT}{\ensuremath{\mathsf{CNOT}}}
\newcommand{\CCNOT}{\ensuremath{\mathsf{CCNOT}}}
\newcommand{\SWAP}{\ensuremath{\mathsf{SWAP}}}
\newcommand{\controlled}[1]{\ensuremath{\mathsf{Controlled}\text{-}#1}}

\newcommand{\free}[1]{\expval{#1}}

\newcommand{\defeq}{\coloneqq}%\stackrel{\text{\tiny def}}{=}}
\newcommand{\eqdef}{\eqqcolon}%\stackrel{\text{\tiny def}}{=}}
\DeclareMathOperator{\vectorize}{vec}
\DeclareMathOperator{\qubits}{qubits}
\DeclareMathOperator{\End}{End}
\DeclareMathOperator*{\argmin}{arg\,min}
\DeclareMathOperator{\Fid}{F}


% Non math
\newcommand{\ie}{i.\,e.}
\newcommand{\Ie}{I.\,e.}
\newcommand{\eg}{e.\,g.}
\newcommand{\Eg}{E.\,g.}

\def\CPP{{C\nolinebreak[4]\raisebox{.15ex}{++}}}

\usepackage{tikz} % TODO is this needed
\usetikzlibrary{arrows,shadows,positioning,fit}
\usetikzlibrary{quantikz}

\usepackage[style=super, sort=use]{glossaries}
\newglossary*{symbols}{List of Symbols}
\makeglossaries

\usepackage{attrib}
\usepackage{enumitem}
\usepackage{qtree}
\usepackage{standalone}
\usepackage{wrapfig}
\usepackage{cleveref}

\newlist{requirements}{enumerate}{10}
\setlist[requirements]{label*=\arabic*.}
\crefname{requirementsi}{requirements}{requirements}
\Crefname{requirementsi}{Requirements}{Requirements}

\newtheorem{theorem}{Theorem}[section]
\newtheorem{definition}[theorem]{Definition}
\newtheorem{example}[theorem]{Example}
\newtheorem{question}[theorem]{Question}

% ******************************************************
% Bibliographies
% ******************************************************
\addbibresource{bib.bib}

% ******************************************************
% Hyphenation
% ******************************************************
%\hyphenation{put special hyphenation here}

% ******************************************************
% GO!GO!GO! MOVE IT!
% ******************************************************
\begin{document}
\frenchspacing
\raggedbottom
\selectlanguage{american}
\pagenumbering{roman}
\pagestyle{plain}
% ******************************************************
% Frontmatter
% ******************************************************
\begin{titlepage}
    \begin{addmargin}[-1cm]{-3cm}
        \begin{center}
            \large

            \hfill

            \vfill

            {
                \color{CTtitle}\spacedallcaps{\myTitle} \\ \bigskip
            }

            \spacedlowsmallcaps{\myName}

            \vfill

            \includegraphics[width=6cm]{img/uwlogo} \\ \medskip

            \myDegree \\
            \myDepartment \\
            \myFaculty \\
            \myInst \\
            \myUni \\ \bigskip

            \myTime

            \vfill

        \end{center}
    \end{addmargin}
\end{titlepage}

\thispagestyle{empty}

\hfill

\vfill

\noindent\myName: \textit{\myTitle,} \myDegree,
\textcopyright\ \myTime

\bigskip

\noindent\spacedlowsmallcaps{Supervisors}: \\
\mySupervisor

\medskip

\noindent\spacedlowsmallcaps{Location}: \\
\myLocation (completed remotely during COVID-19)

\medskip

\noindent\spacedlowsmallcaps{Time Frame}: \\
\myTimeFrame

\cleardoublepage\include{aux/dedication}
\cleardoublepage% ******************************************************
% Abstract
% ******************************************************
%\renewcommand{\abstractname}{Abstract}
\pdfbookmark[1]{Abstract}{Abstract}

\begingroup
\let\clearpage\relax
\let\cleardoublepage\relax
\let\cleardoublepage\relax

\chapter*{Abstract}

This thesis details the problem of quantum circuit compilation.
Starting from the very definition of compile, we introduce many of the ideas needed to understand the main problem of circuit compilation from the very basics.
We cover classical compilers and show how the effort to build effective circuit compilers draws heavily from its classical counterparts.
Upon introducing the formalism of quantum computation, we are able to formulate many of the problems related to circuit compilation in a mathematical language, and detail some of the cutting edge efforts.
We end by showing how circuit compilation is part of a much larger ``quantum stack'' that needs to be created to have effective quantum computers.

\vfill

\endgroup

\vfill

\cleardoublepage% ******************************************************
% Acknowledgments
% ******************************************************
\pdfbookmark[1]{Acknowledgments}{acknowledgments}

\begin{flushright}{\slshape
        They didn't have much trouble \\
        teaching the ape to write poems: \\
        first they strapped him into a chair, \\
        then tied the pencil around his hand \\
        (the paper had already been nailed down). \\
        Then Dr.\ Bluespire leaned over his shoulder \\
        and whispered into his ear: \\
        You look like a god sitting there. \\
        Why don't you try writing something?} \\ \medskip
    --― James Tate
\end{flushright}

\bigskip

\begingroup
\let\clearpage\relax
\let\cleardoublepage\relax
\let\cleardoublepage\relax

\chapter*{Acknowledgments}

Many thanks are in place for the successful completion of this thesis.
First I would like to thank my academic advisor Joel Wallman for the guidance during my bumpy career as a graduate student.
In addition, thank you to the following professors to helping me complete my studies: John Watrous, Achim Kempf, Michael Waite, Brian Ingalls, and Michael Brannan.
Whether it was sharing details about your personal career, asking probing questions, or offering time and having supportive conversation despite not having to: thank you.

Thank you to Joel's research group for helping me deal with Joel's departure: Darian Mclaren, Anthony Chytros, Matthew Graydon, Stefanie Beale, Sam Ferracin, and Joshua Skanes-Norman.
I would also like to thank my many classmates without which remote classes would have been far less interesting and rewarding: Wilson Wu, Chelsea Komlo, Mohammad Ayyash, Nicholas Zutt, and Xiaoran Li.
A big thank you is also in order for Overleaf and in particular John Lees-Miller and Ryan Looney for allowing me to work part time and being extremely flexible with my hours.
It was great to continue working with the team\ldots and to supplement the measly graduate student salary.

Thank you Mom and Dad for letting me live in your house while we endured the brunt of the pandemic.
Thank you Diane for always having my back and being supportive throughout my graduate studies.
Thank you to my friends who were always open to discuss my struggles and triumphs: Kevin (both of you), Rafael, Ana, and Aimee.

\endgroup

\cleardoublepage% ******************************************************
% Table of Contents
% ******************************************************
\pagestyle{scrheadings}
%\phantomsection
\pdfbookmark[1]{\contentsname}{tableofcontents}
\setcounter{tocdepth}{2} % <-- 2 includes up to subsections in the ToC
\setcounter{secnumdepth}{3} % <-- 3 numbers up to subsubsections
\manualmark
\markboth{\spacedlowsmallcaps{\contentsname}}{\spacedlowsmallcaps{\contentsname}}
\tableofcontents
\automark[section]{chapter}
\renewcommand{\chaptermark}[1]{\markboth{\spacedlowsmallcaps{#1}}{\spacedlowsmallcaps{#1}}}
\renewcommand{\sectionmark}[1]{\markright{\textsc{\thesection}\enspace\spacedlowsmallcaps{#1}}}
% ******************************************************
% List of Figures and of the Tables
% ******************************************************
\clearpage
% \pagestyle{empty} % Uncomment this line if your lists should not have any headlines with section name and page number
\begingroup
\let\clearpage\relax
\let\cleardoublepage\relax
% ******************************************************
% List of Figures
% ******************************************************
%\phantomsection
%\addcontentsline{toc}{chapter}{\listfigurename}
\pdfbookmark[1]{\listfigurename}{lof}
\listoffigures

\vspace{8ex}

% ******************************************************
% List of Tables
% ******************************************************
%\phantomsection
%\addcontentsline{toc}{chapter}{\listtablename}
\pdfbookmark[1]{\listtablename}{lot}
\listoftables

\vspace{8ex}
% \newpage

% ******************************************************
% List of Listings
% ******************************************************
%\phantomsection
%\addcontentsline{toc}{chapter}{\lstlistlistingname}
\pdfbookmark[1]{\lstlistlistingname}{lol}
\lstlistoflistings

\vspace{8ex}

% ******************************************************
% Acronyms
% ******************************************************
%\phantomsection
\pdfbookmark[1]{Acronyms}{acronyms}
\markboth{\spacedlowsmallcaps{Acronyms}}{\spacedlowsmallcaps{Acronyms}}
\chapter*{Acronyms}
\begin{acronym}[UMLX]
    \acro{CPU}{Central Processing Unit}
    \acro{IR}{Intermediate Representation}
    \acro{MLIR}{Multi-Level Intermediate Representation}
    \acro{NISQ}{Noisy Intermediate-Scale Quantum}
    \acro{NIST}{National Institute of Standards and Technology}
    \acro{SPAM}{State Preparation and Measurement} % TODO use this
    \acro{CLOPS}{Circuit Layer Operations per Second}
\end{acronym}

\vspace{3ex}

% ******************************************************
% Symbols/Notation
% ******************************************************
\pdfbookmark[1]{List of Symbols}{symbols}
\newglossaryentry{un}{
    type=symbols,
    name={\ensuremath{\U{n}}},
    description={Group of unitary operators or matrices of dimension $n\times n$}
}
\newglossaryentry{sun}{
    type=symbols,
    name={\ensuremath{\SU{n}}},
    description={Group of special unitary operators or matrices of dimension $n\times n$}
}
\newglossaryentry{kleene}{
    type=symbols,
    name={\ensuremath{G^*}},
    description={Kleene star of a finite set $G$}
}
\newglossaryentry{intsn}{
    type=symbols,
    name={\ensuremath{[n]}},
    description={The set of natural numbers up to and including $n$: \ie{} $\qty{0, 1, \ldots, n}$}
}
\newglossaryentry{endV}{
    type=symbols,
    name={\ensuremath{\End{V}}},
    description={The set of endomorphisms, or linear transformations on a vector space $V$.}
}
\printglossaries

\endgroup

% ******************************************************
% Main matter
% ******************************************************
\cleardoublepage
\pagestyle{scrheadings}
\pagenumbering{arabic}
%\setcounter{page}{90}
% use \cleardoublepage here to avoid problems with pdfbookmark

\cleardoublepage

\ctparttext{
    To begin this document we will introduce the notion of a compiler and show the foundational role it plays in our modern computing infrastructure.
    We will cover the main ideas from compilers that are useful for our quest to understand quantum circuit compilers in~\cref{part:backend}.
    We will then switch gears to cover the basics of quantum computation needed to understand the quantum part of our story.
}
\part{Front End}\label{part:frontend}
%************************************************
\chapter{All Things Classical}\label{chap:compilers}
%************************************************

In this chapter we will give a very brief overview of the components of classical computers that will be helpful to further discussions of quantum circuit compilation.
A key component to quantum circuit compilation is the word ``compilation'' whose origins (in computing) date to the early 1950's when electronic digital computers were in their early stages.
Understanding the historical development of compilation and its techniques will provide ideas and tools necessary to solve the new task of quantum circuit compilation.

This chapter is meant to provide the reader with the basics of some computing terminology and ideas.
It is by no means a complete introduction to compilers, nor computer architecture.

\section{What can a computer do?}

If you're reading this, I'm sure you can imagine something your computer is capable of.
Maybe reading this document online, sending messages/email, browsing the internet, writing documents, etc.
These are very high-level operations our computer can perform, but under the hood much more primitive operations are taking place.
It is these primitive operations that we wish to understand, and will have many similarities with modern-day quantum hardware.

A simplified model of computer architecture, known as the von Neumann Architecture (\cref{fig:comparch}) shows what we now call a \ac{CPU} which is the workhorse of the computer.\footnote{At least in this \emph{very simplified} model.}

\begin{figure}[h]
    \centering
    \includestandalone[width=0.8\textwidth]{tikz/arch}
    \caption{von Neumann Architecture}\label{fig:comparch}
\end{figure}

Since the \ac{CPU} is the computational component of the computer, what can \emph{it} do?
Modern \acp{CPU} are built on the \ac{ISA} which means that the \ac{CPU} has a finite set of operations or instructions that it can perform.
Every operation the computer can perform must be built up from these primitive instructions.
Some examples of what these primitive operations might be are:
\begin{itemize}
    \item put a value into memory;
    \item add two values in memory together and store the result in a new location;
    \item perform the bitwise negation on a value;
    \item compute the square root of a value.
\end{itemize}
One can then use these primitives to build up complex functionality that eventually implement the capabilities we know and love (and hate) computers for.

Choosing an \ac{ISA} results in the creation of a \emph{complexity class} which is a collection of problems that can be solved using a polynomial number of primitive instructions/operations.\footnote{Polynomial refers to a function which takes the \emph{size} of the problem input, and returns the number of steps (or time) required to solve the problem.}
In practice, most \acp{ISA} implement the same complexity class, and we denote it by P. % TODO complexity
The formal definition of P is ``decision problems solvable by a deterministic Turing machine in a polynomial amount of time'', but the picture one should have in mind is ``problems for which we have efficient algorithms''.
For more details on complexity classes, and computational complexity in general consult~\cite[Chapter 3]{nielsenchuang} for the material with an eye towards quantum, and~\cite{complexity} for a more detailed exposition.

The \ac{ISA} architecture style has seen major success, but it suffers from the drawback of requiring the programmer to work at the very low-level of machine instructions.
To work at a higher level of abstraction, and hence to have a higher-level of productivity, computer scientists and programmers created new languages which were easier to read, write, and reason about.
This necessitated new languages to be ``translated'' into the instruction set after the code was written.
The software responsible for translating these higher level ideas into a machines instruction set are known as \textbf{compilers}.

\section{Compilers}

While compilers have their origins in the aforementioned translation of higher-level code into lower-level code, they have grown considerably to perform many more tasks.
Before we dive into all of the capabilities of modern compilers, let's take a step back and recall what the word compile means.

Merriam-Webster~\cite{compiledef} defines the word \emph{compile} to mean
\begin{quote}
    to compose out of materials from other documents.
\end{quote}
In the context of programming language compilers,  ``other documents'' might mean the code itself, as well as configuration files and environment variables.
With these materials the lower level machine code is then composed.
This definition is reflected in~\citetitle{dragonbook}\footnote{Colloquially known as ``The Dragon Book'' because of the cover, and likely the most famous book on (classical) compilers. This is also where the logo of the LLVM project originates from which we will discuss in~\cref{sec:llvm}.}\graffito{
    \includegraphics[width=\marginparwidth]{img/dragonbook.png}
    \emph{The Dragon Book}
    \includegraphics[width=\marginparwidth]{img/llvmlogo.png}
    \emph{LLVM Logo}
}~\cite{dragonbook} where the authors introduce compilers through the process of transforming software.
\begin{quotation}
    [B]efore a program can be run, it first must be translated into a form in which it can be executed by a computer.

    The software systems that do this translation are called \emph{compilers}.
\end{quotation}
Hence we can view compilers as a function taking software written at one level of abstraction and bringing it down to a lower level that a computer's \ac{CPU} can understand.
\begin{figure}[ht]
    \centering
    \includestandalone[width=0.8\textwidth]{tikz/compiler}
    \caption{Action of Compiler}\label{fig:compiler}
\end{figure}

The term compiler was first used in the context computers by Grace Hopper in the early 1950's while working on a system that could translate symbolic mathematics into a machine language.
Initially Hopper's new idea was met with resistance as it was thought to be unrealistic.
\begin{quotation}
    I had a running compiler, and nobody would touch it because, they carefully told me, computers could only do arithmetic; they could not do programs.
    It was a selling job to get people to try it.
    I think with any new idea, because people are allergic to change, you have to get out and sell the idea.
    \attrib{Grace Hopper~\cite{hopperquote}}
\end{quotation}
In the end, Hopper succeeded in selling the idea and compilers have become a ubiquitous piece of modern computing infrastructure.
While Hopper's compiler focused solely on code translation, a modern compiler might perform all of line reconstruction, preprocessing, lexical analysis, syntax analysis, semantic analysis, conversion to an \acf{IR}, optimization (and there are many different types!), and finally code generation.
Thankfully we will not need to understand \emph{all} of these parts in full, but rather will focus on \aclp{IR}, optimizations, and code generation.

\paragraph{Resources}
Before jumping into processes that make up a compiler, we will first detail some of the hardware restrictions that compilers must be aware of while performing their job.
Modern digital computers are built on the transistor: a small\footnote{They are today, but they were not always small!} device models a bit (0 or 1) as the absence or flow of electricity.
Since transistors provide the basic building block of the bit, the transistor count is an effective measure of how much memory the computer has.
Hence when a compiler is performing some sort of optimizations, it must be aware of the amount of memory it can make use of.
While modern computers have an abundance of memory, not all memory is created equally.
The \ac{CPU} can talk to the computers long-term storage (hard drive), it's a slow communication that is not ideal to be performing frequently.
Instead, \acp{CPU} have their own (smaller) internal memory which is often referred to as a collection of registers.
These registers provide fast access to variables during computation.
Hence, during the compilation of a program, the compilers knowledge of the target hardware's \ac{CPU} allows the compiler to efficiently use the on-board registers, and make informed decisions as to when to use long-term storage.

The second resource that compilers are often made aware to is the \acp{CPU} ability to run parallelized computations.
The ability to perform multiple instructions at the same time is often taken advantage of in compilers via techniques like loop unrolling.\footnote{\url{https://en.wikipedia.org/wiki/Loop_unrolling}}
However, not all architectures support this mode of optimization, and even if it does, the compiler must be careful to ensure parallelization optimizations do not over-burden registers.

\subsection{Compilation Phases}\label{sec:comp-phases}

As alluded to in the previous section, a compiler has many different responsibilities.
Each responsibility is broken into a separate component so that it can be understood on its own, and later be reused in its own context.
A schematic for this can be seen in~\cref{fig:compilerphases} on page~\pageref{fig:compilerphases} showing the main steps that we will be concerned with in this document.

\paragraph{Syntax Analyzer}
This phase is for ensuring the code is syntactically well formed (that is, that it abides by the specification of the language).
If one is writing code in a binary alphabet with characters \texttt{0} and \texttt{1}, then the ``program'' \texttt{00011} is syntactically valid, while \texttt{1102} is not because a \texttt{2} appears in the code.
Many compilers transform the code into a syntax tree to complete the verification.

\paragraph{Semantic Analyzer}
Now that the code is syntactically valid, we can ensure it has meaning.
This phase usually consists of type checking and scope validation (ensuring the code does not access variables outside of scope).
In many compiled languages the operation \texttt{'hello' * 5} would pass syntax analysis, but fail semantic analysis because a string multiplied by an integer is not a valid operation.\footnote{It is completely valid in other languages like Python, but Python is not a compiled language.}

\begin{wrapfigure}[19]{i}{0.25\textwidth} % TODO tweak lineheight
    \centering
    \includestandalone[width=0.23\textwidth]{tikz/phases}
    \caption{Compiler Phases}\label{fig:compilerphases} % TODO can we fix this caption?
\end{wrapfigure}

\paragraph{Intermediate Code Generator}
The code is now ensured to be well formed and can begin preparation to execute on hardware.
Passing directly to the code generator is possible from here, but the end product will be slower as no optimizations will take place.
Instead, the existing code (or sometimes using the syntax tree created in the previous steps) will be transformed into an \acf{IR}.
This is a mid-level representation of the code in that it is typically thought of as somewhere between the high-level of abstraction of the programming language, and the low-level instruction set.

This is best seen with a simple example.
Suppose we have the following snippet to calculate the final location of a moving object after 5 seconds.
\begin{lstlisting}
    x_final = x_initial + velocity * 5
\end{lstlisting}
Upon transforming this code to an \ac{IR}, it takes on a more basic form.
\begin{lstlisting}
    t1 = inttofloat(5)
    t2 = velocity * t1
    t3 = x_initial + t2
    x_final = t3
\end{lstlisting}
The power here comes from the fact that the \acf{IR} can be language agnostic, and hence many languages can compile into the same \ac{IR}.
This design allows for the use of an optimizer for many languages.

\paragraph{Code Optimizer}
Once the code is in the \ac{IR}, the optimizer will attempt to ``improve'' it using many different methods.
Improve can mean many different things, but usually refers to runtime and memory use.
Optimizations that occur during this step are constant propagation, dead code elimination, removing unnecessary code from loops, and loop unrolling.
Optimizing the above example our code is still ``bulkier'' than originally written, but compressed in comparison to the original \ac{IR}-form.
\begin{lstlisting}
    t1 = velocity * 5.0
    x_final = x_initial + t1
\end{lstlisting}
Here we have skipped the call to \texttt{inttofloat} and instead immediately converted the integer \texttt{5} to the float \texttt{5.0}.
We have also combined two of the steps to reduce the number of temporary variables we have to create and store in memory.
As you can see the task of the optimizer is not only to try and speed up the code, but reduce its memory usage as well.
Some of the other problems the code optimizer must tackle are instruction selection, register allocation, and instruction scheduling all of which have analogs we will see in~\cref{chap:circuit-compilers}.

\paragraph{Code Generator}
Finally we have an optimized \ac{IR} and we can generate code for hardware.
This requires us to know which hardware it is we'd like to run our code on as each chip might have a different \ac{ISA}.
This is a very difficult step as many of the sub-problems that are required to be solved are themselves NP-complete such as register allocation~\cite{register-allocation-NP}. % TODO complexity
Further, generating mathematically optimal machine code has also been shown to be undecidable~\cite{dragonbook}!
Hence this step uses effective heuristics to solve the problem at hand in tractable amounts of time.
Typically this step is broken down into first optimizing the \ac{IR} for the hardware that has been chosen, followed by the actual code generation.
If this occurs the optimizer is typically referred to as a hardware-independent optimizer, and a later stage of optimizations is performed in a hardware-dependent optimizer.
We will see later that the distinct phases of optimization are of crucial importance when compiling quantum circuits.

Again following the above code example, upon code generation we may end with the following generic hardware instruction code.

\begin{minipage}{0.5\textwidth}
    \begin{lstlisting}
    LDF R2, velocity
    MULF R2, R2, #5.0
    LDF R1, x_initial
    ADDF R1, R1, R2
    STF x_final, R1
\end{lstlisting}
\end{minipage}
\begin{minipage}{0.5\textwidth}
    \centering
    \begin{tabular}{rl}
        Function      & Meaning         \\ \toprule
        \texttt{LDF}  & Load float      \\
        \texttt{MULF} & Multiply floats \\
        \texttt{ADDF} & Add floats      \\
        \texttt{STF}  & Store float
    \end{tabular}
    \captionof{table}{Machine Code}\label{fig:machcode}
\end{minipage}
Here anything beginning with \texttt{R} is a register.

The phases described here are often grouped into three larger categories.
The syntax and semantic analysis, as well as the generation of an \ac{IR} fall under the umbrella of ``front end'', the optimizer is the optimizer, and everything else that follows is the ``back end''.
The implications of this design is that an optimizer and backend can be paired with many different front ends as long as the front end can generate the optimizer's preferred \ac{IR} flavor.
\begin{figure}[ht]
    \centering
    \includestandalone[width=0.75\textwidth]{tikz/frontback}
    \caption{Compiler with many front and back ends}\label{fig:compends}
\end{figure}

\subsection{Optimizations}\label{sec:optimizations}

Before moving on to some examples of compilers, its important to understand the separation of concerns in the two types of optimizations we've seen.
The main optimizer we see in~\cref{fig:compilerphases} as ``Code optimizer'' and again the ``Optimizer'' in~\cref{fig:compends} are typically where the majority of optimizations take place in classical compilers and are performed on an \ac{IR}.
One interesting class of examples are peephole optimizations~\cite{classical-peephole}.
These are optimizations that take advantage of small patterns found in code that can be simplified in some way.
Some examples are seen in~\cref{tab:peephole}.
\begin{table}[ht]
    \centering
    \begin{tabular}{p{.5\textwidth}l}
        Instruction                                                      & Optimized Instruction \\ \toprule
        Read value into a register, then immediately store it in memory. & Do nothing            \\
        $a \cdot x + b \cdot x$                                          & $(a + b) \cdot x$     \\
        $x - x$                                                          & $0$                   \\
        $(A^\intercal B^\intercal)^\intercal$                            & $BA$
    \end{tabular}
    \caption{Peephole Optimizations}\label{tab:peephole}
\end{table}
Other examples include dead code elimination, common subexpression elimination, and inlining.
The optimizations done here---usually to the ends of faster runtime and smaller memory use---are performed in the hopes that once the code is compiled into machine code it \emph{will} run faster.
The intuitive optimizations often remove duplication, but many other optimizations that are not so clear take advantage of the commonalities among \ac{CPU} design to produce code that will run faster on any \ac{CPU}.

With an optimized \ac{IR}, and a chosen backend, or hardware, the code can be modified to suit the instruction set, as well as other restrictions the hardware may place on computation.
For example, most \acp{CPU} have a small number of registers, and hence must use them wisely throughout the computation so as to use \emph{all} of them where possible, but not slow down computation by waiting for a register to be available.
Another example is instruction scheduling, where the compiler must figure out an optimal ordering to the computation, again to maximize the \acp{CPU} compute power while not causing bottlenecks.
There are many other examples of hardware-dependent optimizations, but as you might imagine, many require an intimate knowledge of the hardware's particular design.
All this transformation occurs while maintaining the same semantic meaning of the original program.

In summary the first hardware-independent optimization should be thought of as optimizing the implementation theoretically, and the hardware-dependent optimization as ensuring the optimized algorithm runs as fast as possible in its final implementation.
Many more examples of optimizations (both hardware-independent and hardware-dependent) can be found in~\cite[Chapter~8]{compiler-optimizations}.

\subsection{Examples}\label{sec:compiler-examples}

We've now seen what a compiler is and what we typically use it for.
A few examples are in order to help understand how compilers work in the real world, and just how varied they can be.

\begin{description}
    \item[clang:] Short for C Language, this is a compiler frontend for the C/\CPP{} languages. It takes in C/\CPP{} code and produces an LLVM \ac{IR} which we will learn about in~\cref{sec:llvm}. It then lets LLVM handle the rest of the compilation processing.
    \item[Latex:] While perhaps not very obvious, \LaTeX{} is indeed a compiler as it takes high-level formatting code, and produces a lower level representation of what the user wants to typeset. Usually that comes in the form of postscript which is another programming language that is read by printers (hardware) to produce the requested document. Postscript can also be read by PDF readers and browsers which then display content as the author desired (maybe).
    \item[TensorFlow:] TensorFlow is a library for machine learning that has drawn on the design principles of compilers in attempts to speed up and ensure the accuracy of models. Indeed it has a frontend where the user builds their model and compiles it into an \ac{IR} known as HLO \ac{IR} or High Level Operations. Typical optimizations then occur and again using the LLVM compiler infrastructure this code can be brought to many backends such as the browser, mobile, and specialized compute infrastructure (such as Google's \ac{TPU}). This is all before we talk about TensorFlow Quantum which allows for hybrid quantum-classical machine learning models~\cite{tensoflowquantum}.
\end{description}

\section{LLVM}\label{sec:llvm}

The LLVM\footnote{The project, while originally an acronym for Low Level Virtual Machine now goes solely by LLVM. The original name reflects the fact that the compiler targets low-level \ac{IR} code that runs on some theoretical (hence the term virtual) machine. Since the inception virtual machines have come to mean something different, hence the abandonment of the acronym.} project~\cite{llvm} is one of the largest open source compiler projects in existence and much of the compiler architecture we've discussed here come from its design.
The founder of the project Chris Lattner has characterized compilers succinctly in~\cite{lattnerquote} as
\begin{quote}
    the art of allowing humans to think at a level of abstraction that they want to think about.
\end{quote}

As an interesting historical note, once the \ac{ISA} scheme had become commonplace, chip designers began to implement more and more complex instructions on \acp{CPU} so that machine code became higher level.
At the same time, compilers became more popular, especially as their optimizations became more robust, and useful.
This led to a distinction between chip architectures known as \ac{CISC} and \ac{RISC}.
At the time of writing, \ac{CISC} processors are dominant in desktop computers, while \ac{RISC} processors emphasize efficiency and can be found in phones and many other portable computing hardware.
Some examples include Intel's x86 and x64 chips which are built in the \ac{CISC} style, while ARM is major designer of \ac{RISC} chips (including the most recent Apple Bionic A15 chip). % TODO add note about RISC V?
Today \acp{RISC} are sometimes referred to using the backronym ``Relegate Interesting Stuff to the Compiler''.

With the growth of LLVM, developers have pushed the compiler to extend its use to ``heterogeneous hardware''~\cite{mlir}, which already includes new types of computing hardware like \acp{TPU} and could in the future encompass a \ac{QPU}.
This is exciting not only because classical computer designers are beginning to consider quantum technologies as coprocessors, but because the monumental classical computing infrastructure can then be leveraged to aid in the solutions to quantum problems.
With the futurism, hype, and unknowns surrounding quantum technologies, it often seems that fundamentally new and ingenious ideas are needed to forward the field.
Projects such as the above show there are serious possibilities of recycling, or at the very least, learning from what has come before us.

\include{chapters/2-quantum}

\cleardoublepage

\ctparttext{
    The goal of this part is to familiarize the reader with the realities of quantum hardware.
    This means understanding their architecture, strengths, weaknesses, and some of the many measures we have to quanitify their effectiveness.
    With an understanding of the limitations of modern-day hardware we can understand the problem of quantum circuit compilation.
    We will show how the problem is both similar and different from classical compilation and how we can benefit from using existing classical infrastructure.
}
\part{Back End}\label{part:backend}
\include{chapters/3-hardware}
% ***********************************************
\chapter{Circuit Compilers}\label{chap:circuit-compilers}
% ***********************************************

We can now return to the topic of compilers.
It should now be clear that the level of abstraction we work at when designing quantum algorithms (\ie{} quantum circuits possibly with some some classical computation mixed in) is much higher than the capabilities of our current, and likely near-future hardware.
Hence, just as we saw in~\cref{chap:compilers}, we are in need of a tool to translate this description down to a lower level of abstraction that embodies the restrictions of the hardware.
As in~\cref{fig:compilerphases} which detailed the phases of a compiler, there are syntax and semantic analyses that are performed to ensure circuits are well formed, but we will not go any further into this topic here.
The most interesting, and complicated portions of circuit compilation occur in transforming a circuit to an \ac{IR}, optimizing it, and generating machine level instructions.
Na\"{\i}vely this is three phases, but because current quantum hardware is so restrictive this can often be broken down into the following four phases.

\begin{enumerate}
    \item Conversion of quantum algorithm to a \ac{QIR}.
    \item Optimization of the \ac{QIR}.
    \item Compilation of the \ac{QIR} to a specific quantum chip, resulting in an instruction set.
    \item Optimization of the instruction set.
\end{enumerate}
This is reflected in the following diagram.
\begin{figure}[h] % TODO finish diagram
    \centering
    \includestandalone[width=0.8\textwidth]{tikz/qcompiler}
    \caption{Action of Quantum Compiler}\label{fig:quantumcompiler}
\end{figure}

This reflects the structure of a classical compiler very closely in part because the phased approach works well, but as we will see later it suits our needs well for hybrid quantum-classical computations that are expected to be the dominant near-term use of quantum computers.
This approach also allows the design of components to be easily reused just as we saw with classical compiler in~\cref{fig:compends}.
A similar figure can be drawn for some of the many players in the quantum landscape and can be seen in~\cref{fig:optionsq}.
\begin{figure}[ht]
    \centering
    \includestandalone[width=0.8\textwidth]{tikz/frontbackq}
    \caption{Modularity of Quantum Compiler}\label{fig:optionsq}
\end{figure}

One of the benefits of the modular compiler structure seen in~\cref{fig:optionsq} is that once the optimizer is made, backends can be written as new hardware arrive, \emph{and} a backend can be written to take the circuit to a classical \ac{CPU}.
In effect what this provides is an optimized quantum simulator.

Many proposals for a \ac{QIR} are built on top of the LLVM \ac{IR} because of the success it has had in classical computing.
In particular the QIR Alliance~\cite{qir} has been formed in order to formalize a specification for a \ac{QIR} that will describe quantum and classical computation.
This project has already had some success as a \ac{MLIR} has already been made that lowers into the LLVM \ac{IR} in a way that is adherent to the \ac{QIR} specification put forth~\cite{mlirquantum}.
As we will see in~\cref{sec:methods} many near-term applications of quantum computers will use quantum computers as a coprocessor of information, rather than operating independently.
Thus having a unified \ac{IR} that is capable of describing quantum and classical computation is compulsory.
This reinforces the benefits of building a \ac{QIR} on top of an existing \ac{IR}.


\paragraph{Fault Tolerance} % TODO: expand into compiling into logical qubits/gate, then optimizing those gadgets
As we saw in~\cref{sec:ft}, fault tolerance is key method for encoding qubits and gates to prevent the spread of errors in a quantum circuit.
This is done by restricting where entangling gates can be applied.
Thus when compiling a fault tolerant circuit, the compiler needs to understand not only the restrictions that may be in place due to the quantum chips connectivity, but \emph{also} the entangling gate restriction that fault tolerance places on the circuit.
Not only this, but it is hoped that we may also be able to use compilers to take circuits and compile them \emph{into} a fault tolerant form if the quantum chip allows for it.

\begin{example}[Compiling the Toffoli Gate]
    Since most hardware are not capable of 3 qubit operations we must decompose the Toffoli gate into something more manageable.
    This is typically done using \CNOT{}'s, Hadamard's ($H$), and $\pi/8$ ($T$) gates~\cite{nielsenchuang}.
    \begin{equation}
        \begin{quantikz}[column sep=.25cm]
            & \ctrl{1} & \midstick[3,brackets=none]{$=$} \qw & \qw      & \qw      & \qw              & \ctrl{2} & \qw      & \qw      & \qw              & \ctrl{2} & \qw      & \ctrl{1} & \gate{T}         & \ctrl{1} & \qw \\
            & \ctrl{1} & \qw                                 & \qw      & \ctrl{1} & \qw              & \qw      & \qw      & \ctrl{1} & \qw              & \qw      & \gate{T} & \targ{}  & \gate{T^\dagger} & \targ{}  & \qw \\
            & \targ{}  & \qw                                 & \gate{H} & \targ{}  & \gate{T^\dagger} & \targ{}  & \gate{T} & \targ{}  & \gate{T^\dagger} & \targ{}  & \gate{T} & \gate{H} & \qw              & \qw      & \qw
        \end{quantikz}
    \end{equation}
    This is an important decomposition as the \CCNOT{} gate appears in the modular exponentiation problem which is a core part of Shor's factoring algorithm~\cite{shor-encryption}.
    Hence if there are smaller decompositions than shown above that would be ideal as \emph{one} \CCNOT{} gate becomes 14!
    \citeauthor{universal-cnot-u2} show a more compact decomposition of \CCNOT{} using only 3 \CNOT{} gates if the phase of one of the qubits is allowed to change~\cite{universal-cnot-u2}.
    Let $G = R_Y(\frac{\pi}{4})$ in the following circuit. % following https://arxiv.org/pdf/quant-ph/9705009.pdf rather than original paper
    \begin{equation}
        \begin{quantikz}%[column sep=.25cm]
            & \ctrl{1} & \midstick[3,brackets=none]{$\approx$} \qw & \qw                        & \qw      & \qw                        & \ctrl{2} & \qw                       & \qw      & \qw                       & \qw \\
            & \ctrl{1} & \qw                                       & \qw                        & \ctrl{1} & \qw                        & \qw      & \qw                       & \ctrl{1} & \qw                       & \qw \\
            & \targ{}  & \qw                                       & \gate{G^\dagger} & \targ{}  & \gate{G^\dagger} & \targ{}  & \gate{G} & \targ{}  & \gate{G} & \qw
        \end{quantikz}
    \end{equation}
    However the question of ``how many \CNOT{} gates does it take to decompose a \CCNOT{}?'' was answered in~\citeyear{toff3cnot} when it was shown that a true equality preserving decomposition requires a minimum of 6 \CNOT{} gates~\cite{toff3cnot}.\footnote{This result shows that a minimum of 6 \CNOT{} gates must be used, \textbf{if} they are being used. Other decompositions not using \CNOT{} gates might still be more compact.}
\end{example}


\section{Compiling on a ring}\label{sec:ringcomp}

In this section we will see an example that will take us through some of the many difficulties one might face while attempting to come up with a general purpose algorithm/method for compiling quantum circuits.
This example is drawn from~\cite{ring-compilation} with modifications.

To begin, suppose we'd like to run the quantum circuit shown in~\cref{fig:presquish}.
\begin{figure}[ht]
    \centering
    \begin{quantikz}%[row sep=.2cm]
        & \targ{}   & \qw      & \gate{X} & \qw      & \ctrl{1} & \qw      & \qw      & \qw      & \qw       & \targ{}   & \qw       & \qw \\
        & \qw       & \ctrl{2} & \qw      & \ctrl{1} & \targ{}  & \ctrl{1} & \qw      & \gate{H} & \targ{}   & \ctrl{-1} & \qw       & \qw \\
        & \ctrl{-2} & \qw      & \qw      & \targ{}  & \qw      & \targ{}  & \gate{H} & \qw      & \ctrl{-1} & \qw       & \targ{}   & \qw \\
        & \qw       & \targ{}  & \qw      & \qw      & \qw      & \qw      & \qw      & \qw      & \qw       & \qw       & \ctrl{-1} & \qw
    \end{quantikz}
    \caption{Circuit to be compiled}\label{fig:presquish}
\end{figure}
The first step we can take is to compress the diagram into a fewer number of layers.
To do this we group operations on nonoverlapping qubits since they can be performed at the same time.\footnote{This is not always an option as some implementations of quantum hardware (\eg{} trapped ion), and hence the grouping might not be as compact.}
This is vital as decoherence times (\cref{sec:Ttimes}) are so short.
This ``compressed'' version of the circuit is seen in~\cref{fig:precomp}.
\begin{figure}[ht]
    \centering
    \begin{quantikz}
        & \targ{}   & \qw \slice{} & \gate{X} \slice{} & \ctrl{1} \slice{} & \qw \slice{} & \qw \slice{} & \qw \slice{} & \targ{}   & \qw \\
        & \qw       & \ctrl{2}     & \ctrl{1}          & \targ{}           & \ctrl{1} \gategroup[2,steps=3,style={dashed,rounded corners,fill=blue!20, inner xsep=2pt}, background]{}    & \gate{H}     & \targ{}      & \ctrl{-1} & \qw \\
        & \ctrl{-2} & \qw          & \targ{}           & \qw               & \targ{}      & \gate{H}     & \ctrl{-1}    & \targ{}   & \qw \\
        & \qw       & \targ{}      & \qw               & \qw               & \qw          & \qw          & \qw          & \ctrl{-1} & \qw
    \end{quantikz}
    \caption{Circuit after compression}\label{fig:precomp}
\end{figure}

We can now apply a type of ``device independent optimization'' known as ``peephole optimization'' just as we saw in~\cref{sec:optimizations}, using the fact that $\CNOT^2_1\cdot (H \otimes H)\cdot\CNOT^1_2 = H \otimes H$.
This minor optimization, and many others can be found in~\cite{peephole}.
Hence we can drop the two \CNOT{} gates in the blue box to obtain the figure seen in~\cref{fig:predevice}.
\begin{figure}[ht]
    \centering
    \begin{quantikz}
        q_1 & \targ{}   & \qw \slice{} & \gate{X} \slice{} & \ctrl{1} \slice{} & \qw \slice{} & \targ{}   & \qw \\
        q_2 & \qw       & \ctrl{2}     & \ctrl{1}          & \targ{}           & \gate{H}     & \ctrl{-1} & \qw \\
        q_3 & \ctrl{-2} & \qw          & \targ{}           & \qw               & \gate{H}     & \targ{}   & \qw \\
        q_4 & \qw       & \targ{}      & \qw               & \qw               & \qw          & \ctrl{-1} & \qw
    \end{quantikz}
    \caption{Circuit after peephole optimization}\label{fig:predevice}
\end{figure}

To continue with the problem we must now choose hardware we would like to run this circuit on.
As the section title suggest, we will be choosing a qubit network topology of a ring.
\begin{figure}[ht] % TODO make labels Q_i instead of i
    \centering
    \includestandalone[width=0.4\textwidth]{tikz/ring}
    \caption{Ring Topology}\label{fig:ring}
\end{figure}
The first problem we need to tackle is placing the qubits from the circuit onto the ring.
The first slice of the circuit contains \CNOT{}s connecting $q_1 \leftrightarrow q_3$ and $q_2 \leftrightarrow q_4$ so placing them together to prevent additional \SWAP{}s from being added is the first task.
There are many configurations to satisfy this, but only one\footnote{Modulo ring rotations/reflections.} that satisfy the requirements that no \SWAP{} gates are added in the second slice as well!
That mapping is
\begin{align}
    q_1 \to 1 &  & q_2 \to 3 &  & q_3 \to 2 &  & q_4 \to 4.
\end{align}
Hence the first two slices of the circuit can be computed without any additional \SWAP{} gates being added.

Executing the gates in slice 3 however will require a \SWAP{} as qubits $q_1$ and $q_2$ are no longer adjacent.
To make these qubits adjacent we can either swap qubits $q_1$ and $q_3$ or $q_2$ and $q_3$.
Looking ahead to slice 5 we see we need adjacency of $q_1 \leftrightarrow q_2$ and $q_3 \leftrightarrow q_4$.
Swapping $q_1$ and $q_3$ would mean two additional \SWAP{} gates before slice 5, but swapping $q_2$ and $q_3$ leaves the qubits in their desired positions for slice 5.
Hence our compiled circuit in it final form:
\begin{figure}[ht]
    \centering
    \begin{quantikz}
        & \targ{}   & \gate{X}  & \qw           & \ctrl{1} & \qw      & \qw           & \targ{}   & \qw \\
        & \ctrl{-1} & \targ{}   & \gate[swap]{} & \targ{}  & \gate{H} & \gate[swap]{} & \ctrl{-1} & \qw \\
        & \ctrl{1}  & \ctrl{-1} &               & \qw      & \gate{H} &               & \targ{}   & \qw \\
        & \targ{}   & \qw       & \qw           & \qw      & \qw      & \qw           & \ctrl{-1} & \qw
    \end{quantikz}
    \caption{Compiled Circuit}\label{fig:compcirc}
\end{figure}

If the quantum chip has the further restriction that it's network topology is a directed graph and all the edges point clockwise, we can no longer use the typical \SWAP{} decomposition we are used to as in~\cref{eq:cnotswap}.
Instead we must use
\begin{equation}
    \begin{quantikz}
        & \targ{}   & \midstick[2,brackets=none]{$=$} \qw & \gate{H} & \ctrl{1} & \gate{H} & \qw \\
        & \ctrl{-1} & \qw                                 & \gate{H} & \targ{}  & \gate{H} & \qw
    \end{quantikz}
\end{equation}
in~\cref{eq:cnotswap} to decompose \SWAP{} using only \CNOT{} gates that go in one direction.
\begin{equation}
    \begin{quantikz}
        & \gate[swap]{} & \midstick[2,brackets=none]{$=$} \qw & \targ{}   & \gate{H} & \targ{}   & \gate{H} & \targ{}   & \qw \\
        &               & \qw                                 & \ctrl{-1} & \gate{H} & \ctrl{-1} & \gate{H} & \ctrl{-1} & \qw
    \end{quantikz}
\end{equation}
With this addition the compiled circuit begins to grow very quickly (\cref{fig:dirringcomp})
\begin{figure}[ht]
    \centering
    \begin{quantikz}[column sep=.1cm]
        & \gate{H} & \ctrl{1} & \gate{H} & \gate{X} & \qw      & \qw      & \qw      & \qw      & \qw      & \qw      & \ctrl{1} & \qw      & \qw      & \qw      & \qw      & \qw      & \qw      & \gate{H} & \ctrl{1} & \gate{H} & \qw \\
        & \gate{H} & \targ{}  & \qw      & \ctrl{1} & \gate{H} & \ctrl{1} & \gate{H} & \ctrl{1} & \gate{H} & \ctrl{1} & \targ{}  & \gate{H} & \ctrl{1} & \gate{H} & \ctrl{1} & \gate{H} & \ctrl{1} & \gate{H} & \targ{}  & \gate{H} & \qw \\
        & \qw      & \ctrl{1} & \gate{H} & \targ{}  & \gate{H} & \targ{}  & \gate{H} & \targ{}  & \gate{H} & \targ{}  & \qw      & \gate{H} & \targ{}  & \gate{H} & \targ{}  & \gate{H} & \targ{}  & \gate{H} & \ctrl{1} & \gate{H} & \qw \\
        & \qw      & \targ{}  & \qw      & \qw      & \qw      & \qw      & \qw      & \qw      & \qw      & \qw      & \qw      & \qw      & \qw      & \qw      & \qw      & \qw      & \qw      & \gate{H} & \targ{}  & \gate{H} & \qw
    \end{quantikz}
    \caption{Compiled Circuit on Directed Ring}\label{fig:dirringcomp}
\end{figure}

While this was a relatively simple example of some of the tasks a circuit compiler must complete, it did not begin to touch on the problem of gate decomposition or unitary synthesis (\cref{qu:optimalsynthesis}).
In the above example all gates applied were taken to be native to the hardware.


\section{Methods}\label{sec:methods}

Current research on compilation methods can be benchmarked in many ways, and compilation techniques often arise to improve on a given benchmark.
Benchmarks are typically performed with respect to the most prominent compiler, which at the time of writing seems to be that of IBM's Qiskit~\cite{qiskit}.
Just as we saw in~\cref{sec:comp-phases} many of the subproblems required to be solved in classical compilation are NP-complete, or more difficult.
Unfortunately the situation seems as bad in quantum compilation as the problem of assigning logical qubits to physical ones is equivalent to the subgraph isomorphism problem which is known to be NP-complete.
Again finding the optimal number, and position, of \SWAP{} gates is equivalent to another problem known to be at least NP-hard.
Thus, as before we must look to heuristic solutions.

The procedure we encountered in~\cref{sec:ringcomp} was loosely based on methods proposed in~\cite{ring-compilation} where first a circuit is sliced by timesteps, an initial mapping of qubits is made to the connectivity graph, routing gates from the original circuit onto the new architecture is performed, finally ending with gate synthesis for the gates that the quantum chip may not support.\footnote{Followed by peephole optimizations if they are available.}
In~\cite{intprog} considers solely the problem of finding optimal solutions to the qubit assignment, and routing problems.
Despite this being an NP-complete problem the authors make the simplifying assumption that the circuit is already decomposed into one and two qubit gates that are native to the hardware.
The problem is then encoded via a complex integer programming problem, with similarly encoded cost functions such as minimizing the total error rate, minimizing circuit depth, and minimizing crosstalk.
Once encoded, the optimization problem can then be solved by one of the many integer programming libraries.
The authors report a decrease in \CNOT{} gates and higher fidelity when run on real hardware when compared to Qiskit's compiler.

A slightly different approach is taken through~\cite{nisq-comp,nisq-comp2,nisq-comp3} where compilers are designed specifically for \ac{NISQ}-era devices.
In particular~\cite{nisq-comp2,nisq-comp3} use calibration data collected from hardware to inform the compilation process.
This means if a particular qubit has a very high error rate, the compiler attempts to route computation around it, or use it as infrequently as possible.
This allows the compiler to generate circuits optimized for the hardware at particular times of day as calibration data changes intra-day.

Deep reinforcement learning has also made its way into quantum circuit compilation in attempt to perform unitary synthesis~\cite{deepcompile}.
This approach is well-suited for real-time quantum computation where the additional time required to compile a circuit is unavailable and hence a more immediate solution is required.

\paragraph{QAOA}
The \ac{QAOA} is a combinatorial optimization algorithm that is intended to be run on \ac{NISQ}-era devices~\cite{qaoa}.
With the knowledge of a circuits general structure, compilers have been able to take advantage of this fact.
Focusing in this particular problem, a 23\% reduction in gate count, and 53\% reduction in circuit depth was acheived~\cite{qaoa-compiler}.
In the future we might hope to build these problem-specific compilers into a more general purpose one that can diagnose and understand when to use problem-specific compilers on demand.

\paragraph{VQE}
Another hybrid quantum-classical algorithm that has seem much attention due to its near-term applications in quantum chemistry is that of the \ac{VQE}~\cite{vqe,vqe2}.
This algorithm is used to calculate the ground state of a molecular Hamiltonian using a parametrized quantum circuit as a cost function, and the classical compute nodes as an optimizer.
\Eg{} let $\boldsymbol{\theta} \in \R^n$ be a vector of numbers that our circuit $U$ depends on, \ie{} $U: \R^n \to \U{2^m}$ for some number of qubits $m$.
\begin{figure}[ht]
    \centering
    \includestandalone[width=0.7\textwidth]{tikz/vqe}
    \caption{\acs{VQE} Schematic}\label{fig:vqe}
\end{figure}
A compiler specific to this problem has been created, and generalized to other quantum-classical algorithms~\cite{vqe-compiler} leveraging much of the existing infrastructure brought forth by the LLVM project discussed in~\cref{sec:llvm}.
This allows the classical optimizations to be handled by the robust LLVM system, while using new circuit compilation techniques that take advantage of the fact that variational circuits have a particular form.
The structure of variational circuits has been further been taken advantage of by pre-compiling specific blocks of gates which resulted in 1.5--3 times improvement over existing systems~\cite{vqe-partial}.

\paragraph{Crosstalk}
Due to crosstalk's prevalence on nearly all hardware, compilers have been developed to mitigate this problem by utilizing both commutation identities and physical gate timing~\cite{crosstalk-commute,crosstalk-mitigation}. % TODO add picture from either paper

\subsection{Verification}

While retaining the semantic meaning of a circuit is one of the highest priorities during circuit compilation, it is possible it has changed.
Thus, just as chip manufacturers use verification techniques to ensure electronics are built to specification, circuit compilation can also benefit from such techniques.
With smaller circuits, it's possible to ensure the correctness of compilers by simulation, but this is not a scalable approach to due to the inherent complexities in simulating quantum systems.
To this end various methods have been developed such as formal proof~\cite{circuit-verification-formal-proof}, diagrammatic methods~\cite{circuit-verification-diagrammatic}, equivalence checking~\cite{circuit-verification-equivalence-check} and functional verification~\cite{circuit-verification-functional}.
There have also been circuit optimizers written in formal languages like Coq~\cite{coq} using the semantics of matrices to only perform optimizations it has formally verified to be correct~\cite{verified-optimizer}.


\section{Quantum Stack}

In this section we will give an overview of the related technologies to quantum circuit compilation and try and explain how they fit together to form a quantum full stack.
\todo{all these need to be tied together}

\texttt{qcor}~\cite{qcor} is a \CPP{} compiler for hybrid quantum-classical computing built on \texttt{clang} (\CPP{} compiler mentioned in~\cref{sec:compiler-examples}).

Authors in~\cite{qisa} introduce the \texttt{quil} language~\cite{quil} which is intended to be a Quantum \ac{ISA} suitable for program analysis and compilation.

OpenQASM~\cite{openqasm2,openqasm3} aims to be a lower-level programming language that can handle hybrid quantum-classical computations.

\texttt{staq}~\cite{staq} is a full stack toolkit that can be used for compilation of OpenQASM.

ScaffCC is a a scalable compilation and analysis framework based on LLVM~\cite{scaffcc,scaffcc2}.

\include{chapters/5-conclusion}

% ******************************************************
% Back matter
% ******************************************************
\cleardoublepage\include{aux/bibliography}
\cleardoublepage%*******************************************************
% Declaration
%*******************************************************
\pdfbookmark[0]{Declaration}{declaration}
\chapter*{Declaration}
\thispagestyle{empty}
This thesis consists of material all of which I authored.

I understand that my thesis may be made electronically available to the public.
\bigskip

\noindent\textit{\myLocation, \myTime}

\smallskip

\begin{flushright}
    \begin{tabular}{m{5cm}}
        \\ \hline
        \centering\myName \\
    \end{tabular}
\end{flushright}

\cleardoublepage\pagestyle{empty}

\hfill

\vfill

\pdfbookmark[0]{Colophon}{colophon}
\section*{Colophon}
This document was typeset using the typographical look-and-feel \texttt{classicthesis} developed by Andr\'e Miede and Ivo Pletikosić.
The style was inspired by Robert Bringhurst's seminal book on typography ``\emph{The Elements of Typographic Style}'' and is available at:
\begin{center}
    \url{https://bitbucket.org/amiede/classicthesis/}
\end{center}

Hermann Zapf's \emph{Palatino} and \emph{Euler} type faces are used and the ``typewriter'' text is typeset in \emph{Bera Mono}.



\end{document}
