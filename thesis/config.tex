\PassOptionsToPackage{utf8}{inputenc}
\usepackage{inputenc}

\PassOptionsToPackage{T1}{fontenc}
\usepackage{fontenc}


% ********************************************************************
% 1. Configure classicthesis for your needs here
% ********************************************************************
\PassOptionsToPackage{
    drafting=true,
    eulerchapternumbers=true,
    floatperchapter=true,
    beramono=true,
    palatino=true,
}{classicthesis}


% ********************************************************************
% 2. Personal data and user ad-hoc commands (insert your own data here)
% ********************************************************************
\newcommand{\myTitle}{Quantum Circuit Compilation From The Ground Up\xspace}
\newcommand{\mySubtitle}{\xspace}
\newcommand{\myDegree}{Master of Mathematics (MMath)\xspace}
\newcommand{\myName}{Nathaniel Stemen\xspace}
% \newcommand{\myProf}{Someone else TODO\xspace}
\newcommand{\mySupervisor}{Joel Wallman\xspace}
\newcommand{\myFaculty}{Faculty of Mathematics\xspace}
\newcommand{\myDepartment}{Department of Applied Mathematics\xspace}
\newcommand{\myUni}{University of Waterloo\xspace}
\newcommand{\myInst}{Institute for Quantum Computing\xspace}
\newcommand{\myLocation}{Seattle, WA\xspace}
\newcommand{\myTime}{April 2022\xspace}
\newcommand{\myTimeFrame}{September 2020---April 2022}
\newcommand{\myVersion}{\classicthesis}

% ********************************************************************
% 3. Loading some handy packages
% ********************************************************************

% ********************************************************************
% Packages with options that might require adjustments
% ********************************************************************
\PassOptionsToPackage{american}{babel}
\usepackage{babel}

\usepackage{csquotes}

\PassOptionsToPackage{
    style=alphabetic,
    maxnames=5,
    backref=true
}{biblatex}
\usepackage{biblatex}

\PassOptionsToPackage{fleqn}{amsmath} % float equations towards left side
\usepackage{amsmath}
\usepackage{amssymb}
\newtheorem{theorem}{Theorem}[section]
\newtheorem{definition}[theorem]{Definition}
\newtheorem{example}[theorem]{Example}
\newtheorem{question}[theorem]{Question}

% ********************************************************************
% General useful packages
% ********************************************************************
\usepackage{graphicx}
\usepackage{scrhack} % fix warnings when using KOMA with listings package
\usepackage{xspace} % to get the spacing after macros right
\PassOptionsToPackage{printonlyreused,smaller}{acronym}
\usepackage{acronym} % TODO change to acro package
\usepackage{physics}
\usepackage{mathtools}
\usepackage{stmaryrd}
\DeclarePairedDelimiter\implement{\llbracket}{\rrbracket}
\usepackage{xfrac}


% ********************************************************************
% 4. Setup floats: tables, (sub)figures, and captions
% ********************************************************************
\usepackage{tabularx} % TODO change to booktabs
\setlength{\extrarowheight}{3pt} % increase table row height
\usepackage{subfig}


% ********************************************************************
% 5. Setup code listings
% ********************************************************************
\usepackage{listings}
%\lstset{emph={trueIndex,root},emphstyle=\color{BlueViolet}}%\underbar} % for special keywords
\lstset{language=[LaTeX]Tex,%C++,
    morekeywords={PassOptionsToPackage,selectlanguage},
    keywordstyle=\color{RoyalBlue},%\bfseries,
    basicstyle=\small\ttfamily,
    %identifierstyle=\color{NavyBlue},
    commentstyle=\color{Green}\ttfamily,
    stringstyle=\rmfamily,
    numbers=none,%left,%
    numberstyle=\scriptsize,%\tiny
    stepnumber=5,
    numbersep=8pt,
    showstringspaces=false,
    breaklines=true,
    %frameround=ftff,
    %frame=single,
    belowcaptionskip=.75\baselineskip
    %frame=L
}


% ********************************************************************
% 6. Last calls before the bar closes
% ********************************************************************
\usepackage{classicthesis}


% ********************************************************************
% Fine-tune hyperreferences (hyperref should be called last)
% ********************************************************************
\hypersetup{%
    % COLOR ***********
    colorlinks=true,
    linktocpage=true,
    urlcolor=CTurl, linkcolor=CTlink, citecolor=CTcitation,
    % urlcolor=Black, linkcolor=Black, citecolor=Black, % for printing
    % IDK TBH *********
    breaklinks=true,
    pageanchor=true,
    bookmarksnumbered=true, bookmarksopen=true, bookmarksopenlevel=1,
    % PDF META ********
    pdftitle={\myTitle},
    pdfauthor={\textcopyright\ \myName, \myUni},
    pdfsubject={Quantum Circuit Compilation},
    pdfkeywords={quantum information, quantum computation, quantum circuits, compilers},
    pdfcreator={pdfLaTeX},
    pdfproducer={LaTeX with hyperref and classicthesis},
}


% ********************************************************************
% Setup autoreferences (hyperref and babel)
% ********************************************************************
% There are some issues regarding autorefnames
% http://www.tex.ac.uk/cgi-bin/texfaq2html?label=latexwords
% you have to redefine the macros for the
% language you use, e.g., american, ngerman
% (as chosen when loading babel/AtBeginDocument)
% ********************************************************************
\makeatletter
\@ifpackageloaded{babel}%
{%
    \addto\extrasamerican{%
        \renewcommand*{\figureautorefname}{Figure}%
        \renewcommand*{\tableautorefname}{Table}%
        \renewcommand*{\partautorefname}{Part}%
        \renewcommand*{\chapterautorefname}{Chapter}%
        \renewcommand*{\sectionautorefname}{Section}%
        \renewcommand*{\subsectionautorefname}{Section}%
        \renewcommand*{\subsubsectionautorefname}{Section}%
    }%
    % Fix to getting autorefs for subfigures right (thanks to Belinda Vogt for changing the definition)
    \providecommand{\subfigureautorefname}{\figureautorefname}%
}{\relax}
\makeatother
